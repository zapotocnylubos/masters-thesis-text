%% This is the ctufit-thesis example file. It is used to produce theses
%% for submission to Czech Technical University, Faculty of Information Technology.
%%
%% This is version 1.3.10, built 27. 2. 2025.
%% 
%% Get the newest version from
%% https://gitlab.fit.cvut.cz/theses-templates/FITthesis-LaTeX
%%
%%
%% Copyright 2024, Tomas Novacek
%% Copyright 2021, Eliska Sestakova and Ondrej Guth
%%
%% This work may be distributed and/or modified under the
%% conditions of the LaTeX Project Public License, either version 1.3
%% of this license or (at your option) any later version.
%% The latest version of this license is in
%%  https://www.latex-project.org/lppl.txt
%% and version 1.3 or later is part of all distributions of LaTeX
%% version 2005/12/01 or later.
%%
%% This work has the LPPL maintenance status `maintained'.
%%
%% The current maintainer of this work is Tomas Novacek.
%% Alternatively, submit bug reports to the tracker at
%% https://gitlab.fit.cvut.cz/theses-templates/FITthesis-LaTeX/issues
%%
%%

% arara: xelatex
% arara: biber
% arara: xelatex
% arara: xelatex

%%%%%%%%%%%%%%%%%%%%%%%%%%%%%%%%%%%%%%%%%
% CLASS OPTIONS
% language: czech/english/slovak
% thesis type: bachelor/master/dissertation
% colour: bw for black&white OR no option for default colour scheme
% electronic (oneside) or printed (twoside), twoside is default
% paragraph - if passed, this optional argument sets paragraphs as the deepest level of headers, styles it, numbers it and adds it to Table of Content. Use with care! Normally, it is considered unwise to use it, since its too deep.
%%%%%%%%%%%%%%%%%%%%%%%%%%%%%%%%%%%%%%%%%
\documentclass[czech,master,unicode,oneside]{ctufit-thesis}

%%%%%%%%%%%%%%%%%%%%%%%%%%%%%%%%%%
% FILL IN THIS INFORMATION
%%%%%%%%%%%%%%%%%%%%%%%%%%%%%%%%%%
\ctufittitle{Porovnání Frama-C a Stainless} % replace with the title of your thesis
\ctufitauthorfull{Bc. Luboš Zápotočný} % replace with your full name (first name(s) and then family name(s) / surname(s)) including academic degrees
\ctufitauthorsurnames{Zápotočný} % replace with your surname(s) / family name(s)
\ctufitauthorgivennames{Luboš} % replace with your first name(s) / given name(s)
\ctufitsupervisor{doc.\,RNDr.\,Dušan Knop,\,Ph.D.} % replace with name of your supervisor/advisor (include academic degrees)
\ctufitdepartment{Katedra teoretické informatiky} % replace with the department of your defence
\ctufityear{2025} % replace with the year of your defence
\ctufitdeclarationplace{Praze} % replace with the place where you sign the declaration
\ctufitdeclarationdate{\today} % replace with the date of signature of the declaration
\ctufitabstractCZE{
Formální verifikace se zabývá dokazováním správnosti programů na základě matematických metod a logiky. Ověřené programy díky tomu poskytují záruku, že pracují v souladu s předem definovanou specifikací za všech okolností, a výrazně tak snižují riziko kritických chyb, které by tradiční testování mohlo přehlédnout.

Tato diplomová práce se věnuje teoretickým aspektům formální verifikace a studiu dostupných nástrojů pro automatizované dokazování (SMT řešičů), jako jsou Alt-Ergo, CVC4/5 a Z3, a jejich integraci v prostředích Frama-C (pro jazyk C) a Stainless (pro jazyk Scala). Zároveň porovnává přístupy obou nástrojů na praktickém příkladu formálně verifikované implementace datové struktury AVL stromu, přičemž zvláštní důraz klade na rozbor rozdílů v dokazování pomocí těchto dvou frameworků.
}
\ctufitabstractENG{
Formal verification focuses on proving program correctness using mathematical methods and logic. Verified programs therefore guarantee that they behave according to a predefined specification under all circumstances, markedly reducing the risk of critical errors that traditional testing could overlook.

This thesis explores the theoretical aspects of formal verification and investigates the available automated-proving tools (SMT solvers) such as Alt-Ergo, CVC4/5, and Z3, along with their integration within the Frama-C environment (for the C language) and Stainless (for Scala). It also compares the approaches of these two toolchains through a practical case study: a formally verified implementation of the AVL-tree data structure, with special emphasis on analysing the differences in proof development across the two frameworks.
}
\ctufitkeywordsCZE{formalní verifikace, SMT, Frama-C, Stainless}
\ctufitkeywordsENG{formal verification, SMT, Frama-C, Stainless}
%%%%%%%%%%%%%%%%%%%%%%%%%%%%%%%%%%
% END FILL IN
%%%%%%%%%%%%%%%%%%%%%%%%%%%%%%%%%%

%%%%%%%%%%%%%%%%%%%%%%%%%%%%%%%%%%
% CUSTOMIZATION of this template
% Skip this part or alter it if you know what you are doing.
%%%%%%%%%%%%%%%%%%%%%%%%%%%%%%%%%%

\RequirePackage{iftex}[2020/03/06]
\iftutex % XeLaTeX and LuaLaTeX
    \RequirePackage{ellipsis}[2020/05/22] %ellipsis workaround for XeLaTeX
\else
    \errmessage{Only compilation with XeLaTeX or LuaLaTeX is allowed}
    \stop
\fi

% hyperlinks
\hypersetup{
    pdfpagelayout=TwoPageRight,
    colorlinks=false,
    allcolors=decoration,
    pdfborder={0 0 0.1}
}

% uncomment the following to hide all hyperlinks
%\hypersetup{hidelinks}

% uncomment the following to change the colour of all hyperlinks to CTU blue
%\hypersetup{allbordercolors=decoration}

\RequirePackage{pdfpages}[2020/01/28]

%%%%%%%%%%%%%%%%%%%%%%%%%%%%%%%%%%
% CUSTOMIZATION of this template END
%%%%%%%%%%%%%%%%%%%%%%%%%%%%%%%%%%


%%%%%%%%%%%%%%%%%%%%%%
% DEMO CONTENTS SETTINGS
% You may choose to modify this part.
%%%%%%%%%%%%%%%%%%%%%%
\usepackage{dirtree}
\usepackage{lipsum,tikz}
\usepackage[style=iso-numeric]{biblatex}
\addbibresource{text/bib-database.bib}
\usepackage{xurl}
\usepackage{listings} % typesetting of sources
\usepackage{minted}
\usepackage{csquotes}
\usepackage{float} % for [H] placement of figures

%%%%%%%%%%%%%%%%%%%%%%
% DEMO CONTENTS SETTINGS END
%%%%%%%%%%%%%%%%%%%%%%

\begin{document} 
\frontmatter\frontmatterinit % do not remove these two commands

\thispagestyle{empty}\maketitle\thispagestyle{empty}\cleardoublepage % do not remove these four commands

\includepdf[pages={1-}]{zapotlub-assignment.pdf} % replace this file with your thesis assignment generated from ProjectsFIT

\imprintpage % do not remove this command
\stopTOCentries
%%%%%%%%%%%%%%%%%%%%%%
% list of other contents END
%%%%%%%%%%%%%%%%%%%%%%

%%%%%%%%%%%%%%%%%%%
% ACKNOWLEDGMENT
% FILL IN / MODIFY
% This is a place to thank people for helping you. It is common to thank your supervisor.
%%%%%%%%%%%%%%%%%%%
\begin{acknowledgmentpage}
	Chtěl bych poděkovat především sit amet, consectetuer adipiscing elit. Curabitur sagittis hendrerit ante. Class aptent taciti sociosqu ad litora torquent per conubia nostra, per inceptos hymenaeos. Cras pede libero, dapibus nec, pretium sit amet, tempor quis. Sed vel lectus. Donec odio tempus molestie, porttitor ut, iaculis quis, sem. Suspendisse sagittis ultrices augue.
\end{acknowledgmentpage} 
%%%%%%%%%%%%%%%%%%%
% ACKNOWLEDGMENT END
%%%%%%%%%%%%%%%%%%%


%%%%%%%%%%%%%%%%%%%
% DECLARATION
% FILL IN / MODIFY
%%%%%%%%%%%%%%%%%%%
% INSTRUCTIONS
% ENG: choose one of approved texts of the declaration. DO NOT CREATE YOUR OWN. Find the approved texts at https://courses.fit.cvut.cz/SFE/download/index.html#_documents (document Declaration for FT in English)
% CZE/SLO: Vyberte jedno z fakultou schvalenych prohlaseni. NEVKLADEJTE VLASTNI TEXT. Schvalena prohlaseni najdete zde: https://courses.fit.cvut.cz/SZZ/dokumenty/index.html#_dokumenty (prohlášení do ZP)
\begin{declarationpage}
Prohlašuji, že jsem předloženou práci vypracoval samostatně a že jsem uvedl veškeré použité
informační zdroje v souladu s Metodickým pokynem o dodržování etických principů při přípravě
vysokoškolských závěrečných prací.

Beru na vědomí, že se na moji práci vztahují práva a povinnosti vyplývající ze zákona č. 121/2000 Sb.,
autorského zákona, ve znění pozdějších předpisů, zejména skutečnost, že České vysoké učení
technické v Praze má právo na uzavření licenční smlouvy o užití této práce jako školního díla podle §
60 odst. 1 citovaného zákona.

Prohlašuji, že jsem v průběhu příprav a psaní závěrečné práce použil nástroje umělé
inteligence. Vygenerovaný obsah jsem ověřil. Stvrzuji, že jsem si vědom, že za obsah
závěrečné práce plně zodpovídám.
\end{declarationpage}
%%%%%%%%%%%%%%%%%%%
% DECLARATION END
%%%%%%%%%%%%%%%%%%%

\printabstractpage % do not remove this command

%%%%%%%%%%%%%%%%%%%
% SUMMARY
% FILL IN / MODIFY
% OR REMOVE ENTIRELY (upon agreement with your supervisor)
% (appropriate to remove in most theses)
%%%%%%%%%%%%%%%%%%%
% \begin{summarypage}
% \section*{Summary section}
% 
% \lipsum[1][1-8]
% 
% \section*{Summary section}
% 
% \lipsum[2][1-6]
% 
% \section*{Summary section}
% 
% \lipsum[3]
% 
% \section*{Summary section}
% 
% \lipsum[2]
% 
% \section*{Summary section}
% 
% \lipsum[1][1-8] Lorem lorem lorem.
% \end{summarypage}
%%%%%%%%%%%%%%%%%%%
% SUMMARY END
%%%%%%%%%%%%%%%%%%%

\tableofcontents % do not remove this command
%%%%%%%%%%%%%%%%%%%%%%
% list of other contents: figures, tables, code listings, algorithms, etc.
% add/remove commands accordingly
%%%%%%%%%%%%%%%%%%%%%%
\listoffigures % list of figures
\begingroup
\let\clearpage\relax
\listoftables % list of tables
\thectufitlistingscommand
\endgroup

%%%%%%%%%%%%%%%%%%%
% ABBREVIATIONS
% FILL IN / MODIFY
% OR REMOVE ENTIRELY
% List the abbreviations in lexicography order.
%%%%%%%%%%%%%%%%%%%
\chapter{\thectufitabbreviationlabel}
	
\begin{tabular}{rl}
DFA & Deterministic Finite Automaton\\
FA & Finite Automaton\\
LPS & Labelled Prüfer Sequence\\
NFA & Nondeterministic Finite Automaton\\
NPS & Numbered Prüfer Sequence\\
XML & Extensible Markup Language\\
XPath & XML Path Language\\
XSLT & eXtensible Stylesheet Language Transformations\\
W3C & World Wide Web Consortium
\end{tabular}
%%%%%%%%%%%%%%%%%%%
% ABBREVIATIONS END
%%%%%%%%%%%%%%%%%%%
\resumeTOCentries
\mainmatter\mainmatterinit % do not remove these two commands
%%%%%%%%%%%%%%%%%%%
% THE THESIS
% MODIFY ANYTHING BELOW THIS LINE
%%%%%%%%%%%%%%%%%%%

\chapter*{Úvod}
\addcontentsline{toc}{chapter}{Úvod}
\markboth{Úvod}{Úvod}

Tohle je můj úvod práce.

\setcounter{page}{1}

\chapter{SMT}

Satisfiability Modulo Theories (SMT) je rozšíření SAT, které se zabývá rozhodováním o tom,
zda existuje přiřazení proměnných, které splňuje danou logickou formuli.
SMT se používá v různých oblastech, jako je verifikace softwaru, analýza programů a automatizované dokazování.

\section{Standard SMT-LIB}
\section{SMT řešiče}
\subsection{Eager přístup}
\subsection{Lazy přístup}
\subsection{Alt-Ergo}
\subsection{CVC}
\subsection{Z3}

\chapter{Hoareova logika}
\label{ch:hoareova-logika}

Formální systém označovaný jako Hoareova logika byl navržen a popsán
britským matematikem a informatikem C. A. R. Hoarem v roce 1969~\cite{Hoare1969}.
Jedná se o formální systém pro popis a analýzu počítačových programů,
který pro důkaz správnosti programů používá předpoklady (preconditions) a následky (postconditions).

Program popisujeme pomocí Hoareovy trojice

\begin{equation*}
    \{P\} \  Q \  \{R\}
\end{equation*}

kde $P$ jsou předpoklady popisující stav systému před provedením programu,
$Q$ je daný program (příkaz či sekvence příkazů)
a $R$ jsou následky, které popisují stav systému po provedení tohoto programu.
Takto popsaný systém čteme následovně:
``Při splněných předpokladech $P$ budou po provedení programu $Q$ zaručeny následky $R$``.

Hoare v roce 1969 definoval trojici jako $P \ \{ Q \} \  R$,
ale v současnosti se častěji setkáme se zápisem $\{ P \} \  Q \ \{ R \}$.

Předpoklady a následky jsou vyjádřeny jako logické formule,
které popisují vlastnosti proměnných programu v určitém okamžiku.
Následující příklad reprezentuje program, který předpokládá na vstupu nezápornou hodnotu v proměnné $x$
a po provedení programu $Q$ bude zajištěno, že hodnota proměnné $x$ bude kladná (ostře větší než 0).

\begin{equation*}
    \{ x \geq 0 \} \  x \coloneqq x + 1 \  \{ x > 0 \}
\end{equation*}

Předpoklad může být také prázdný, což znamená, že program nevyžaduje žádné speciální podmínky pro spuštění
a formálně tuto situci lze zapsat jako $\{ true \} \  Q \  \{ R \}$.
Pokud je předpoklad prázdný, předpokládáme, že program může být spuštěn
v libovolném stavu systému nebo s libovolnými hodnotami proměnných.
Takový program může vypadat následovně:

\begin{equation*}
    \{ true \} \  x \coloneqq 10 \  \{ x > 0 \}
\end{equation*}

% TODO: zminit, ze toto je validni program, ale casto nam tento vysledek nic nerika a je potreba
% TODO: jeste pridat podminky na hodnoty promennych v ruznem case atd...

\section{Axiom přiřazení}
\label{sec:hoare-axiom-prirazeni}

Axiom přiřazení je základním pravidlem Hoareovy logiky a nedílnou součástí každého počítačového programu.
Přiřazení je operace obecně zapsaná ve tvaru

\begin{equation*}
    x \coloneqq f
\end{equation*}

kde $x$ je proměnná a $f$ je výraz bez vedlejších efektů (side effect),
který ale může obsahovat proměnnou $x$, například $x \coloneqq x + 1$.

Chceme zajistit, že jakékoli tvrzení $T$ platné o $f$
je také platné o hodnotě proměnné $x$ po provedení přiřazení.
Označíme si $T[x \leftarrow f]$ tvrzení $T$ ve kterém výskyt proměnné $x$ nahradíme výrazem $f$ (substituce).

Poté můžeme definovat axiom přiřazení jako

\begin{equation*}
    \{ T[x \leftarrow f] \} \  x \coloneqq f \  \{ T \}
\end{equation*}

Tento zápis říká, že pokud je pravdivé tvrzení $T$, ve kterém
je výskyt proměnné $x$ nahrazen výrazem $f$, pak po provedení přiřazení
je tvrzení $T$ pravdivé také pro proměnnou $x$.
Jedná se tedy o šablonu (schéma) axiomu,
kterou lze použít pro libovolné tvrzení $T$, proměnnou $x$ a výraz $f$.

% TODO: The assignment axiom proposed by Hoare does not apply when more than one name may refer to the same stored value. For example,
% TODO: end of https://en.wikipedia.org/wiki/Hoare_logic#Assignment_axiom_schema
% TODO: toto ma mozna konsekvence do Frama-C pri volne pametoveho modelu a proc
% TODO: nektere tvrzeni neprojdou na pointerech a rekurzivnich strukturach

\section{Pravidlo důsledku}
\label{sec:hoare-pravidlo-dusledku}

Pravidlo důsledku (rule of consequence) je metoda umožnující
tvorbu nových logických tvrzení z již existujících a dokázaných tvrzení.
Základní aplikací této metody je pravidlo \textbf{rozšíření předpokladu}
a pravidlo \textbf{zůžení následku}.
Předpokládejme platnosti $\{ P \} \  Q \  \{ R \}$.

Máme-li rozšířený předpoklad $P'$, pro který platí

\begin{equation*}
    P' \implies P
\end{equation*}

můžeme říci, že

\begin{equation*}
    \{ P' \} \  Q \  \{ R \}
\end{equation*}

je také platné tvrzení.

Máme-li zúžený následek $R'$ následku $R$, pro který platí

\begin{equation*}
    R \implies R'
\end{equation*}

můžeme říci, že $\{ P \} \  Q \  \{ R' \}$ je také platné tvrzení.

\section{Pravidlo skládání}
\label{sec:hoare-pravidlo-skladani}

Pravidlo skládání (rule of composition) je pravidlo, které umožňuje
skládat více příkazů do jednoho složeného příkazu a používat je v Hoareově logice.

Máme-li dva příkazy $Q_1$ a $Q_2$, které splňují následující tvrzení

\begin{equation*}
    \{ P_1 \} \  Q_1 \  \{ R_1 \}
\end{equation*}

a

\begin{equation*}
    \{ R_1 \} \  Q_2 \  \{ R_2 \}
\end{equation*}

můžeme říci, že složený příkaz $Q_1; Q_2$ splňuje následující tvrzení

\begin{equation*}
    \{ P_1 \} \  Q_1; Q_2 \  \{ R_2 \}
\end{equation*}

kde $;$ je operátor sekvence příkazů, který říká, že příkaz $Q_1$ bude proveden před příkazem $Q_2$.

Toto pravidlo umožňuje vytvářet složitější příkazy z několika jednodušších příkazů.
Zároveň je možné rozmyslet, že na sekvenci příkazů $Q_1, Q_2, \cdots, Q_n$
lze aplikovat stejné pravidlo skládání, které jsme použili pro dva příkazy
pomocí asociativity operátoru $;$ a závorek. A tedy platí, že

\begin{equation*}
    \{ P \} \  Q_1; Q_2; \  \cdots ; Q_n \  \{ R \}
\end{equation*}

je ekvivalentní s

\begin{equation*}
    \{ P \} \  (Q_1; (Q_2; \  \cdots (Q_{n-1}; Q_n))) \  \{ R \}
\end{equation*}

\section{Pravidlo iterace}
\label{sec:hoare-pravidlo-iterace}

Základním stavebním blokem počítačového programu je cyklus.
V Hoareově logice je cyklus reprezentován pomocí pravidla iterace (rule of iteration)
a využívá $while$ cyklus, který je v programovacích jazycích běžně dostupný a je definován následovně:

\begin{equation*}
    while \  B \  do \  Q
\end{equation*}

kde $Q$ je tělo cyklu, které se v každé iteraci provádí, dokud je podmínka $B$ pravdivá.

Dále definujeme invariant cyklu $I$, který nám pomůže
při rozhodování o správnost cyklu a bude popisovat následky provedení cyklu.
Invariant cyklu je logické tvrzení, které musí být pravdivé před vstupem do cyklu,
po ukončení každé iterace cyklu a také po ukončení cyklu.

Formálně můžeme zapsat podmínky pro invariant cyklu $I$ jako

\begin{equation*}
    I \land (\{ B \} \  Q \  \{ I \})
\end{equation*}

První část konjunkce popisuje, že invariant cyklu musí být pravdivý před vstupem do cyklu.
Druhá část konjunkce popisuje, že invariant cyklu musí být pravdivý po provedení těla cyklu $Q$,
pokud se cyklus spustil (podmínka $B$ byla pravdivá).

\begin{remark}
    Invariant cyklu je nezávislý na počtu provedených iterací.
\end{remark}

Pokud cyklus neprovedl žádnou iteraci a zároveň máme zaručeno,
že invariant cyklu $I$ byl pravdivý před vstupem do cyklu,
triviálně platí, že invariant cyklu $I$ je pravdivý i po ukončení cyklu.
Zároveň platí, že cyklus se neprovedl, protože podmínka $B$ byl nepravdivá,

Pokud cyklus provedl alespoň jednu iteraci a zároveň máme zaručeno,
že invariant cyklu $I$ je pravdivý po ukončení těla cyklu $Q$,
znamená to, že invariant cyklu $I$ je pravdivý i po ukončení poslední iterace cyklu.
Zároven platí, že podmínka $B$ je po ukončení cyklu nepravdivá, jinak by cyklus pokračoval v provádění další iterace.

Formálně lze tedy konstrukci $while$ cyklu pomocí Hoareovy logiky zapsat jako

\begin{equation*}
    I \land \{ B \} \  Q \  \{ I \} \implies \{ I \} \  while \  B \  do \  Q \  \{ \neg B \land I \}
\end{equation*}
\chapter{Metoda nejslabšího předpokladu}
\label{ch:metoda-nejslabsiho-predpokladu}

Metoda nejslabšího předpokladu (weakest precondition, WP) je metoda,
používaná k algoritmickému dokazování správnosti programů pomocí Hoareovy logiky.
Formálně tuto metodu popsal E. W. Dijkstra v roce 1975 a navazuje na práci Hoareho~\cite{Dijkstra1975}.

Připomeneme, že Hoareova trojice je trojice $ \{ P \} \ Q \ \{ R \} $,
kde $P$ je předpoklad, $Q$ je příkaz a $R$ je následek.
Předpoklad $P$ je logický výraz, který musí být pravdivý před provedením příkazu $Q$.
Metoda nejslabšího předpokladu se snaží najít nejslabší předpoklad $WP$, pro který platí

\begin{equation*}
    \{ WP \} \ Q \ \{ R \}
\end{equation*}

a zároveň pro každý jiný předpoklad $P$, pro který by platilo, že

\begin{equation*}
    \{ P \} \ Q \ \{ R \}
\end{equation*}

musí také platit, že

\begin{equation*}
    P \implies WP
\end{equation*}

Tedy, že $WP$ je nejslabší předpoklad pro příkaz $Q$ a následek $R$.

\begin{remark}
    Nejslabší předpoklad $WP$ je (logicky) unikátní pro daný příkaz $Q$ a následek $R$.
\end{remark}

Pokud by existoval jiný kandidát $WP'$ pro nejslabší předpoklad,
poté z definice platí, že

\begin{equation*}
    WP' \implies WP
\end{equation*}

a také, že

\begin{equation*}
    WP \implies WP'
\end{equation*}

Tedy musí platit, že $WP$ a $WP'$ jsou (logicky) ekvivalentní.

Nejslabší předpoklad $WP$ definujeme pomocí transformační funkce $wp$
s parametry $Q$ a $R$ následovně:

\begin{equation*}
    WP = wp(Q, R)
\end{equation*}

Následující kapitoly představují základní pravidla pro algoritmický výpočet nejslabšího předpokladu.
Výpočet začíná vždy od následku $R$ a postupně (od konce k začátku)
analyzuje příkazy $Q_n, Q_{n-1}, \cdots, Q_1$ a aplikuje na ně specifická pravidla.
Výsledkem je nejslabší předpoklad $WP$ pro sekvenci příkazů $Q_1; Q_2; \cdots; Q_n$,
který následně použijeme pro formální důkaz správnosti programu.

\section{Pravidlo přiřazení}
\label{sec:pravidlo-prirazeni}

Výpočet nejslabšího předpokladu pro přiřazení proměnné $x$ hodnoty $E$ je
definován následovně:

\begin{equation*}
    wp(x \coloneqq E, R) = R[x \mapsto E]
\end{equation*}

kde $R[x \mapsto E]$ je substituce výsktytu proměnné $x$ hodnotou $E$ v $R$.

Například pro příkaz $x \coloneqq x + 3$ a následek $x > 0$ dostáváme:

\begin{align*}
    wp(x \coloneqq x + 3, x > 0) & = \\
                                 & = x + 3 > 0 \\
                                 & = x > -3
\end{align*}

Nejslabší předpoklad pro příkaz $x \coloneqq x + 3$, který zaručuje, že
následek $x > 0$ bude pravdivý, je tedy $x > -3$.

\section{Pravidlo sekvence}
\label{sec:pravidlo-sekvence}

Pravidlo sekvence napomáhá k určení nejslabšího předpokladu pro sekvenci příkazů.
Máme-li dva příkazy $Q_1$ a $Q_2$, které splňují následující tvrzení

\begin{equation*}
    \{ P_1 \} \  Q_1 \  \{ R_1 \}
\end{equation*}

a

\begin{equation*}
    \{ R_1 \} \  Q_2 \  \{ R_2 \}
\end{equation*}

můžeme říci, že nejslabší předpoklad pro sekvenci příkazů $Q_1; Q_2$ je

\begin{equation*}
    wp(Q_1; Q_2, R) = wp(Q_1, wp(Q_2, R))
\end{equation*}

Podobně jako v pravidle skládání u Hoareovy logiky z kapitoly~\ref{sec:hoare-pravidlo-skladani}
můžeme rozmyslet výpočet nejslabšího předpokladu pro sekvenci příkazů $Q_1, Q_2, \cdots, Q_n$
pomocí asociativity operátoru $;$ a rekurzivního výpočtu nejslabšího předpokladu.

Například pro příkaz $x \coloneqq x + 3; y \coloneqq x + 2$ a následek $y > 0$ dostáváme:

\begin{align*}
    wp(x \coloneqq x + 3; y \coloneqq x + 2, y > 0) & = \\
                                                     & = wp(x \coloneqq x + 3, wp(y \coloneqq x + 2, y > 0)) \\
                                                     & = wp(x \coloneqq x + 3, x + 2 > 0) \\
                                                     & = x + 3 + 2 > 0 \\
                                                     & = x > -5
\end{align*}

Nejslabší předpoklad pro příkaz $x \coloneqq x + 3; y \coloneqq x + 2$, který zaručuje, že
následek $y > 0$ bude pravdivý, je tedy $x > -5$.

\section{Pravidlo podmínky}
\label{sec:pravidlo-podminky}

Pravidlo podmínky popisuje výpočet nejslabšího předpokladu pro podmínkový příkaz ve tvaru:

\begin{equation*}
    \textbf{if} \ B \ \textbf{then} \ Q_T \ \textbf{else} \ Q_F
\end{equation*}

kde $B$ je podmínka, $Q_T$ je příkaz, který se provede, pokud je podmínka $B$ pravdivá,
a $Q_F$ je příkaz, který se provede, pokud je podmínka $B$ nepravdivá.

Pravidlo podmínky je definováno následovně:

\begin{align*}
    wp(\textbf{if} & \ B \ \textbf{then} \ Q_T \ \textbf{else} \ Q_F, R) = \\
                   & = (B \implies wp(Q_T, R)) \land (\neg B \implies wp(Q_F, R))
\end{align*}

Tedy nejslabší předpoklad pro podmínkový příkaz je logická konjunkce dvou implikací.
Nejslabší předpoklad totiž musí zahrnout oba možné stavy podmínky $B$, případ, kdy je $B$ pravdivá ale také případ, kdy je $B$ nepravdivá.

Pokud je podmínka $B$ pravdivá, použijeme nejslabší předpoklad pro příkaz $Q_T$, tedy $wp(Q_T, R)$.
Pokud není pravdivá ($\neg B$), použijeme nejslabší předpoklad pro příkaz $Q_F$, tedy $wp(Q_F, R)$.

Například pro příkaz $if \ x > 0 \ then \ y \coloneqq x + 3 \ else \ y \coloneqq x - 3$
a následek $y > 0$ dostáváme:

\begin{align*}
    wp(if & \ x > 0 \ then \ y \coloneqq x + 3 \ else \ y \coloneqq x - 3, y > 0) = \\
          & = (x > 0 \implies wp(y \coloneqq x + 3, y > 0)) \land (\neg (x > 0) \implies wp(y \coloneqq x - 3, y > 0)) \\
          & = (x > 0 \implies x + 3 > 0) \land (\neg (x > 0) \implies x - 3 > 0) \\
          & = (x > 0 \implies x > -3) \land (\neg (x > 0) \implies x > 3)
\end{align*}

Použitím pravidla implikace

\begin{align*}
    (P \implies A) \land (\neg P \implies B) & \iff (P \land A) \lor (\neg P \land B)
\end{align*}

můžeme původní výraz zjednodušit následovně:

\begin{align*}
    (x > 0 & \implies x > -3) \land (\neg (x > 0) \implies x > 3) = \\
           & = (x > 0 \land x > -3) \lor (\neg (x > 0) \land x > 3) \\
           & = (x > 0 \land x > -3) \lor (x \leq 0 \land x > 3) \\
           & = (x > 0 \land x > -3) \lor (false) \\
           & = x > 0 \land x > -3 \\
           & = x > -3
\end{align*}

Výpočet nejslabšího předpokladu pro příkaz $if \ x > 0 \ then \ y \coloneqq x + 3 \ else \ y \coloneqq x - 3$,
který zaručuje, že následek $y > 0$ bude pravdivý, je tedy $x > -3$.
Poslední zjednodušení je možné provést, protože $x > 0$ je silnější předpoklad než $x > -3$,
platí že $x > 0 \implies x > -3$.

Zároveň jsme při výpočtu zjistili, že negativní část ($else$) podmínky nemá vliv na výpočet nejslabšího předpokladu,
protože neexistuje žádný předpoklad, který by v případě, že $x \leq 0$ a $y \coloneqq x - 3$, zaručoval, že $y > 0$.

% TODO: navazat treba na Frama-c, ze pokud dokazeme takovyto priklad, tak vsechny
% TODO: i nelogicke priklady po tom jsou podminene pravdive, nehlede na to, ze
% TODO: ze nemaji treba splnitelnost = assert \false

\chapter{Frama-C}
\label{ch:frama-c}

V oblasti formální verifikace softwaru hrají významnou roli nástroje umožňující
statickou analýzu zdrojového kódu a deduktivní dokazování správnosti těchto programů.
Mezi ustálené platformy zaměřené na analýzu programů napsaných v jazyce C patří prostředí Frama\mbox{-}C,
které poskytuje užitečné nástroje a možnost rozšíření o moduly a pluginy~\cite{FCKernelMaroneze2024}.
Frama\mbox{-}C je open-source nástroj, který byl vyvinut na výzkumném ústavu CEA-LIST (Commissariat à l'Énergie Atomique et aux Énergies Alternatives)
a první verze byla vydána v roce 2008.

Jádro Frama\mbox{-}C je napsáno v jazyce OCaml a je postaveno jako modulární framework
navržený s cílem usnadnit aplikaci pokročilých technik formální analýzy nad programy v jazyce C
pomocí integrace různých analytických nástrojů ve formě pluginů~\cite{FCPluginDevSignoles2024}.
Některé z těchto pluginů jsou obsaženy přímo v jádře Frama\mbox{-}C, zatímco další jsou dostupné jako externí moduly.
Moduly obsažené v jádře Frama\mbox{-}C zahrnují například deduktivní analyzátor WP (Weakest Precondition)
založený na teoretickém základu popsaném v kapitole~\ref{ch:metoda-nejslabsiho-predpokladu},
RTE (Run-Time Error) pro detekci chyb při běhu programu, který si představíme v kapitole~\ref{sec:frama-c-rte},
nebo například EVA (Evolving Value Analysis) pro analýzu hodnot proměnných v průběhu vykonávání programu.

Klíčovým prvkem Frama\mbox{-}C je definice a podpora specifikačního jazyka ACSL (ANSI/ISO C Specification Language),
který umožňuje uživatelům vyjadřovat vlastnosti a specifikace programů v jazyce C pomocí specifikačních komentářů.
Tyto komentáře anotují kód a poskytují informace pro statickou analýzu a deduktivní dokazování.
Návrh a implementace ACSL byly inspirovány podobným standardem pro jazyk Java, známým jako JML (Java Modeling Language)~\cite{ACSLSpec}.
ACLS bude podrobněji představen v kapitole~\ref{sec:acsl}.

Prostředí Frama\mbox{-}C je distribuováno jako program pro příkazový řádek a také jako grafické uživatelské rozhraní (GUI).
GUI poskytuje uživatelsky přívětivé prostředí pro interakci s celým ekosystémem Frama\mbox{-}C a hlavně s nainstalovanými pluginy.
Grafické prostředí je dostupné hlavně pro operační systémy Linux a Windows.
Pro uživatele operačního systému macOS je k dispozici pouze příkazová řádka.

Organizace stojící za vývojem Frama\mbox{-}C distribuuje mimo jiné také Docker image, které obsahují všechny potřebné závislosti
pro běh Frama\mbox{-}C a základních pluginů společně s několika SMT řešičemi, jako je Alt-Ergo, CVC4 a Z3.
Také distribuují GUI variantu imagů a je tedy velmi snadné spustit Frama\mbox{-}C GUI pomocí Dockeru na libovolném operačním systému,
jako je například macOS nebo Windows.
Tyto image obsahují minimalistické desktopové prostředí, Frama\mbox{-}C GUI a VNC aplikaci
umožnující připojení k desktopovému prostředí pomocí webového prohlížeče~\cite{FCDockerGUIMaroneze2021}.


\section{ACSL (ANSI/ISO C Specification Language)}
\label{sec:acsl}

ACLS je specifikační jazyk pro jazyk C, který umožňuje uživatelům
vyjadřovat vlastnosti a specifikace programů pomocí anotací v podobě speciálních komentářů ve zdrojovém kódu.

Anotace lze zapsat jako jednořádkový komentář \texttt{//@ ...} nebo víceřádkový komentář \texttt{/*@ ... */}.
Tyto anotace jsou umístěny přímo ve zdrojovém kódu a vypadají například jako v ukázce~\ref{list:acsl-example}.

\begin{listing}[H]
    \begin{minted}{C}
    /*@ requires x > 0;
        assigns \result;
        ensures \result == x + 2;
    */
    int increment(int x) {
        x = x + 1;
        //@ assert x > 1;
        return x + 1;
    }
    \end{minted}
    \caption{Ukázka anotace funkce v jazyce C pomocí ACSL}
    \label{list:acsl-example}
\end{listing}

V návaznosti na kapitolu~\ref{ch:metoda-nejslabsiho-predpokladu}, kde jsme se seznámili
s metodou nejslabšího předpokladu, si nyní ukážeme, jakým způsobem lze pomocí ACSL
vyjádřit vlastnosti, předpoklady a závěry pro funkce v jazyce C.

V ukázce~\ref{list:acsl-example} je uvedena funkce \texttt{increment}, která
definuje předpoklad a závěř pomocí klíčových slov \texttt{requires} a \texttt{ensures}.
Zjednodušená Hoaerova trojice pro příklad z kódu~\ref{list:acsl-example} vypadá následovně:

\begin{equation*}
    \{ x > 0 \} \ (x = x + 1; return \  x + 1) \ \{ result == x + 2 \}
\end{equation*}

Frama\mbox{-}C před spuštěním analýzy převede zdrojový kód do mezi-interpretace (intermediate representation)
nazývané \texttt{CIL} (C Intermediate Language)~\cite{Necula2002CIL}.
Frama\mbox{-}C používá vlastní verzi \texttt{CIL}, která je založena na původní verzi, kterou vytvořil George Necula.
Od roku 2016 tato původní verze již není udržována, ale Frama\mbox{-}C stále podporuje vlastní verzi.

Výhodou vlastního předzpracování a použití mezi-interpretace je,
že Frama\mbox{-}C může analyzovat zdrojový kód efektivně
a nainstalované pluginy mohou pracovat s touto abstraktní reprezentací
a nebo ji dokonce upravovat či transformovat~\cite{FCKernelMaroneze2024}.

TODO: pointery a validita, separace

TODO: kvantifikátory, predikáty, logické funkce

TODO: loop invarianty, loop varianty

TODO: ghost konstrukty


\section{RTE (Run-Time Error)}
\label{sec:frama-c-rte}


\section{WP (Weakest Precondition)}
\label{sec:frama-c-wp}


\section{Paměťové modely}
\label{sec:frama-c-mem}

Paměťové modely jsou důležitou součástí analýzy programů,
protože umožňují definovat, jakým způsobem se programy chovají
vzhledem k paměti a jakým způsobem manipulují s daty.

Paměťové modely popisují způsob popisu operace na

\subsection{Hoaerův paměťový model}
\label{subsec:hoaeruv-pametovy-model}

Nejjednodušší paměťový model je Hoareův paměťový model,
ve kterém se operace s pamětí provádějí pomocí přiřazení do logických proměnných.
Jedná se o aplikaci metody Single Static Assignment (SSA),
která zajišťuje, že každá proměnná má přiřazenou hodnotu pouze jednou.

V tomto modelu jsou proměnné a jejich hodnoty v různých časech reprezentovány
pomocí několika různých logických proměnných, na které se odkazujeme pomocí
ACLS anotace \texttt{\textbackslash at}.

Například, pro následující program~\ref{list:ssa-example}:

\begin{listing}[H]
    \begin{minted}{C}
    void calc(int x) {
        x++;
        x += 2;
        //@ assert x >= \at(x, Pre) + 3;
    }
    \end{minted}
    \caption{Ukázka Single Static Assignment v Frama-C}
    \label{list:ssa-example}
\end{listing}

můžeme pomocí Frama\mbox{-}C nahlédnout na interní reprezentaci paměťového modelu
a proměnných pomocí příkazu \texttt{frama-c -wp -wp-no-let -wp-print ssa.c}.
Výstupem je následující kód:

\begin{listing}[H]
    \begin{minted}{text}
    Goal Assertion (file ssa.c, line 4):
    Assume {
      Type: is_sint32(x_1) /\ is_sint32(x)
            /\ is_sint32(x_2) /\ is_sint32(x_3).
      Have: x_3 = x.
      Have: (1 + x_3) = x_2.
      Have: (2 + x_2) = x_1.
    }
    Prove: (3 + x) <= x_1.
    Prover Qed returns Valid
    \end{minted}
    \caption{Ukázka SSA v Frama-C}
    \label{list:ssa-frama-c}
\end{listing}

V příkladu~\ref{list:ssa-frama-c} vidíme, že při každém přiřazení do proměnné $x$ vzniká nová logická proměnná.
Proměnné jsou číslované ve vzestupném pořadí a číslují se v obráceném pořadí, než v jakém byly přiřazeny.
Tedy poslední přiřazení do proměnné $x$ je reprezentováno proměnnou $x_1$,
předposlední přiřazení $x_2$ a tak dále.
Samostatná logická proměnná $x$ reprezentuje hodnotu proměnné $x$ před prvním přiřazením,
tedy na začátku funkce a jedná se o hodnotu proměnné $x$, kterou získáváme voláním \texttt{\textbackslash at(x, Pre)}.

Anotaci \texttt{\textbackslash at(x, L)} můžeme využít pro jakékoliv místo v programu anotované jako $L$.
Návěští (label) $L$ může být například začátek funkce, začátek cyklu nebo libovolné místo v programu, které je ve zdrojovém kódu explicitně označeno.
Uživatelem definovaná návěští jsou standardní součástí jazyka C a Frama\mbox{-}C je používá pro specifikiaci časových bodů v programu.

Například v následujícím programu~\ref{list:label-example}:

\begin{listing}[H]
    \begin{minted}{C}
    void calc(int x) {
        L0:
        x++;
        L1:
        x += 2;
        //@ assert x >= \at(x, L0) + 3;
    }
    \end{minted}
    \caption{Ukázka uživatelského návěští v Frama-C}
    \label{list:label-example}
\end{listing}

jsou explicitně definovaná dvě návěští \texttt{L0} a \texttt{L1},
na která je možné se odkazovat pomocí anotace \texttt{\textbackslash at(x, L0)} a \texttt{\textbackslash at(x, L1)}.

Frama\mbox{-}C poskytuje implicitní návěští, která jsou automaticky generována,
konrkrétně jsou k dispozici tyto návěští~\cite{ACSLSpec}:

% TODO: doplnit popis k os

\begin{itemize}
    \item \textbf{Pre}
    \begin{itemize}
        \item Použitelné pouze v anotacích příkazů, nikoliv v kontraktech funkcí.
        \item Odkazuje na stav před začátkem funkce, ve které se anotace nachází.
    \end{itemize}

    \item \textbf{Post}
    \begin{itemize}
        \item Použitelné pouze v kontraktech funkcí v klauzulích \texttt{assigns} a \texttt{ensures}.
        \item Odkazuje na stav po ukončení funkce, ve které se anotace nachází.
    \end{itemize}

    \item \textbf{Here}
    \begin{itemize}
        \item Použitelné v anotacích příkazů i kontraktech funkcí.
        \item U klauzulí \texttt{requires}, \texttt{assumes}, \texttt{assigns}, \texttt{frees}, \texttt{decreases}, \texttt{terminates} odkazuje na \texttt{Pre} stav.
        \item U klauzulí \texttt{ensures}, \texttt{allocates} a při ukončení s výjimkou odkazuje na \texttt{Post} stav.
        \item V ostatních klauzulích odkazuje na aktualní stav v místě, kde se anotace nachází.
    \end{itemize}

    \item \textbf{LoopEntry}
    \begin{itemize}
        \item Použitelné v anotacích cyklů.
        \item Odkazuje na stav před provedením prvním iterace cyklu.
    \end{itemize}

    \item \textbf{LoopCurrent}
    \begin{itemize}
        \item Použitelné v anotacích cyklů.
        \item Odkazuje na stav před provedením aktuální iterace cyklu.
    \end{itemize}

    \item \textbf{Old}
    \begin{itemize}
        \item Použitelné v anotacích příkazů i kontraktech funkcí.
        \item U klauzulí \texttt{assigns} a \texttt{ensures} odkazuje na stav před provedením funkce.
        \item Oproti \texttt{Pre} lze \texttt{Old} použít i v kontraktech funkcí.
        \item ACSL definuje \texttt{\textbackslash old(x)} jako syntaktický cukr pro \texttt{\textbackslash at(x, Old)}.
    \end{itemize}
\end{itemize}

\subsection{Typový paměťový model}

\subsection{Referenční paměťový model}

\subsection{Problémy paměťových modelů}

\chapter{Stainless}

% --genc does not support recursive data types = src/stainless/LinkedList.scala:41:12: Cons and other recursive types are not supported


\chapter*{Závěr}
\addcontentsline{toc}{chapter}{Závěr}
\markboth{Závěr}{Závěr}

Tohle je můj závěr práce.



\appendix\appendixinit % do not remove these two commands

\include{text/appendix} % include `appendix.tex' from `text/' subdirectory

\backmatter % do not remove this command

\printbibliography % print out the BibLaTeX-generated bibliography list

\chapter{Obsah příloh}
% Contents of the attachment

	\dirtree{%
		.1 /.
		.2 readme.txt\DTcomment{stručný popis obsahu média}.
		.2 exe\DTcomment{adresář se spustitelnou formou implementace}.
		.2 src.
		.3 impl\DTcomment{zdrojové kódy implementace}.
		.3 thesis\DTcomment{zdrojová forma práce ve formátu \LaTeX{}}.
		.2 text\DTcomment{text práce}.
		.3 thesis.pdf\DTcomment{text práce ve formátu PDF}.
	}
 % include `medium.tex' from `text/' subdirectory

\end{document}
