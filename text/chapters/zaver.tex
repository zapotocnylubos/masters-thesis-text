\chapter*{Závěr}
\addcontentsline{toc}{chapter}{Závěr}
\markboth{Závěr}{Závěr}

Tato práce se zaměřila na důkladný popis nástrojů pro formální verifikaci programů,
které se zaměřují na různé programovací jazyky a programovací paradigmata.
Byl popsán základní princip fungování SMT řešičů společně se standardizovaným jazykem SMT-LIB\@.
Poté byl představen teoretický základ ve formě Hoareovy logiky a metody nejslabšího předpokladu,
které jsou základem pro deduktivní verifikaci programů.

Jedním z cílů bylo představit metody pro verifikaci dynamických rekurzivních datových struktur,
reprezentované na příkladu AVL stromu.
Pro vysvětlení problematiky důkazů těchto datových struktur
bylo prostředí Frama\mbox{-}C prozkoumáno do větší hloubky,
včetně architektury paměťových modelů a způsobu reprezentace oddělené paměti.
Bylo možné představit problematické oblasti,
které v současnosti zamezují jednoduchému popisu těchto datových struktur.
Zároveň byla popsána metoda kombinující axiomatický přístup a princip oddělené paměti,
které dohromady při správném použití umožňují verifikaci těchto struktur
bez zavedení velkých a složitých axiomů.
Tato kombinovaná metoda také umožňila začlenění a používání alokačních funkcí (například \texttt{malloc}),
což je velmi málo popisovaný problém v literatuře zabývající se Frama\mbox{-}C\@.

Stainless a jeho přístup k verifikaci dynamických datových struktur
byl bezproblémový a umožnil vcelku jednoduchou verifikaci AVL stromu.
Použité principy v důkazu této datové struktury jsou převoditelné
i na jiné datové struktury.

Závěřem je také vhodné zmínit, že dostupná literatura popisující prostředí Frama\mbox{-}C
je v současnosti bohatší než literatura popisující Stainless.
I přesto byla práce se Stainless příjemná a dokumentace dostačující.
