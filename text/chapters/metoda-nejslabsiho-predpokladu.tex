\chapter{Metoda nejslabšího předpokladu}
\label{ch:metoda-nejslabsiho-predpokladu}

Metoda nejslabšího předpokladu (weakest precondition, WP) je metoda,
používaná k algoritmickému dokazování správnosti programů pomocí Hoareovy logiky.
Formálně tuto metodu popsal E. W. Dijkstra v roce 1975 a navazuje na práci Hoareho~\cite{Dijkstra1975}.

Připomeneme, že Hoareova trojice je trojice $ \{ P \} \ Q \ \{ R \} $,
kde $P$ je předpoklad, $Q$ je příkaz a $R$ je následek.
Předpoklad $P$ je logický výraz, který musí být pravdivý před provedením příkazu $Q$.
Metoda nejslabšího předpokladu se snaží najít nejslabší předpoklad $WP$, pro který platí

\begin{equation*}
    \{ WP \} \ Q \ \{ R \}
\end{equation*}

a zároveň pro každý jiný předpoklad $P$, pro který by platilo, že

\begin{equation*}
    \{ P \} \ Q \ \{ R \}
\end{equation*}

musí také platit, že

\begin{equation*}
    P \implies WP
\end{equation*}

Tedy, že $WP$ je nejslabší předpoklad pro příkaz $Q$ a následek $R$.

\begin{remark}
    Nejslabší předpoklad $WP$ je (logicky) unikátní pro daný příkaz $Q$ a následek $R$.
\end{remark}

Pokud by existoval jiný kandidát $WP'$ pro nejslabší předpoklad,
poté z definice platí, že

\begin{equation*}
    WP' \implies WP
\end{equation*}

a také, že

\begin{equation*}
    WP \implies WP'
\end{equation*}

Tedy musí platit, že $WP$ a $WP'$ jsou (logicky) ekvivalentní.

Nejslabší předpoklad $WP$ definujeme pomocí transformační funkce $wp$
s parametry $Q$ a $R$ následovně:

\begin{equation*}
    WP = wp(Q, R)
\end{equation*}

Následující kapitoly představují základní pravidla pro algoritmický výpočet nejslabšího předpokladu.
Výpočet začíná vždy od následku $R$ a postupně (od konce k začátku)
analyzuje příkazy $Q_n, Q_{n-1}, \cdots, Q_1$ a aplikuje na ně specifická pravidla.
Výsledkem je nejslabší předpoklad $WP$ pro sekvenci příkazů $Q_1; Q_2; \cdots; Q_n$,
který následně použijeme pro formální důkaz správnosti programu.

\section{Pravidlo přiřazení}
\label{sec:pravidlo-prirazeni}

Výpočet nejslabšího předpokladu pro přiřazení proměnné $x$ hodnoty $E$ je
definován následovně:

\begin{equation*}
    wp(x \coloneqq E, R) = R[x \mapsto E]
\end{equation*}

kde $R[x \mapsto E]$ je substituce výsktytu proměnné $x$ hodnotou $E$ v $R$.

Například pro příkaz $x \coloneqq x + 3$ a následek $x > 0$ dostáváme:

\begin{align*}
    wp(x \coloneqq x + 3, x > 0) & = \\
                                 & = x + 3 > 0 \\
                                 & = x > -3
\end{align*}

Nejslabší předpoklad pro příkaz $x \coloneqq x + 3$, který zaručuje, že
následek $x > 0$ bude pravdivý, je tedy $x > -3$.

\section{Pravidlo sekvence}
\label{sec:pravidlo-sekvence}

Pravidlo sekvence napomáhá k určení nejslabšího předpokladu pro sekvenci příkazů.
Máme-li dva příkazy $Q_1$ a $Q_2$, které splňují následující tvrzení

\begin{equation*}
    \{ P_1 \} \  Q_1 \  \{ R_1 \}
\end{equation*}

a

\begin{equation*}
    \{ R_1 \} \  Q_2 \  \{ R_2 \}
\end{equation*}

můžeme říci, že nejslabší předpoklad pro sekvenci příkazů $Q_1; Q_2$ je

\begin{equation*}
    wp(Q_1; Q_2, R) = wp(Q_1, wp(Q_2, R))
\end{equation*}

Podobně jako v pravidle skládání u Hoareovy logiky z kapitoly~\ref{sec:hoare-pravidlo-skladani}
můžeme rozmyslet výpočet nejslabšího předpokladu pro sekvenci příkazů $Q_1, Q_2, \cdots, Q_n$
pomocí asociativity operátoru $;$ a rekurzivního výpočtu nejslabšího předpokladu.

Například pro příkaz $x \coloneqq x + 3; y \coloneqq x + 2$ a následek $y > 0$ dostáváme:

\begin{align*}
    wp(x \coloneqq x + 3; y \coloneqq x + 2, y > 0) & = \\
                                                     & = wp(x \coloneqq x + 3, wp(y \coloneqq x + 2, y > 0)) \\
                                                     & = wp(x \coloneqq x + 3, x + 2 > 0) \\
                                                     & = x + 3 + 2 > 0 \\
                                                     & = x > -5
\end{align*}

Nejslabší předpoklad pro příkaz $x \coloneqq x + 3; y \coloneqq x + 2$, který zaručuje, že
následek $y > 0$ bude pravdivý, je tedy $x > -5$.

\section{Pravidlo podmínky}
\label{sec:pravidlo-podminky}

Pravidlo podmínky popisuje výpočet nejslabšího předpokladu pro podmínkový příkaz ve tvaru:

\begin{equation*}
    \textbf{if} \ B \ \textbf{then} \ Q_T \ \textbf{else} \ Q_F
\end{equation*}

kde $B$ je podmínka, $Q_T$ je příkaz, který se provede, pokud je podmínka $B$ pravdivá,
a $Q_F$ je příkaz, který se provede, pokud je podmínka $B$ nepravdivá.

Pravidlo podmínky je definováno následovně:

\begin{align*}
    wp(\textbf{if} & \ B \ \textbf{then} \ Q_T \ \textbf{else} \ Q_F, R) = \\
                   & = (B \implies wp(Q_T, R)) \land (\neg B \implies wp(Q_F, R))
\end{align*}

Tedy nejslabší předpoklad pro podmínkový příkaz je logická konjunkce dvou implikací.
Nejslabší předpoklad totiž musí zahrnout oba možné stavy podmínky $B$, případ, kdy je $B$ pravdivá ale také případ, kdy je $B$ nepravdivá.

Pokud je podmínka $B$ pravdivá, použijeme nejslabší předpoklad pro příkaz $Q_T$, tedy $wp(Q_T, R)$.
Pokud není pravdivá ($\neg B$), použijeme nejslabší předpoklad pro příkaz $Q_F$, tedy $wp(Q_F, R)$.

Například pro příkaz $if \ x > 0 \ then \ y \coloneqq x + 3 \ else \ y \coloneqq x - 3$
a následek $y > 0$ dostáváme:

\begin{align*}
    wp(if & \ x > 0 \ then \ y \coloneqq x + 3 \ else \ y \coloneqq x - 3, y > 0) = \\
          & = (x > 0 \implies wp(y \coloneqq x + 3, y > 0)) \land (\neg (x > 0) \implies wp(y \coloneqq x - 3, y > 0)) \\
          & = (x > 0 \implies x + 3 > 0) \land (\neg (x > 0) \implies x - 3 > 0) \\
          & = (x > 0 \implies x > -3) \land (\neg (x > 0) \implies x > 3)
\end{align*}

Použitím pravidla implikace

\begin{align*}
    (P \implies A) \land (\neg P \implies B) & \iff (P \land A) \lor (\neg P \land B)
\end{align*}

můžeme původní výraz zjednodušit následovně:

\begin{align*}
    (x > 0 & \implies x > -3) \land (\neg (x > 0) \implies x > 3) = \\
           & = (x > 0 \land x > -3) \lor (\neg (x > 0) \land x > 3) \\
           & = (x > 0 \land x > -3) \lor (x \leq 0 \land x > 3) \\
           & = (x > 0 \land x > -3) \lor (false) \\
           & = x > 0 \land x > -3 \\
           & = x > -3
\end{align*}

Výpočet nejslabšího předpokladu pro příkaz $if \ x > 0 \ then \ y \coloneqq x + 3 \ else \ y \coloneqq x - 3$,
který zaručuje, že následek $y > 0$ bude pravdivý, je tedy $x > -3$.
Poslední zjednodušení je možné provést, protože $x > 0$ je silnější předpoklad než $x > -3$,
platí že $x > 0 \implies x > -3$.

Zároveň jsme při výpočtu zjistili, že negativní část ($else$) podmínky nemá vliv na výpočet nejslabšího předpokladu,
protože neexistuje žádný předpoklad, který by v případě, že $x \leq 0$ a $y \coloneqq x - 3$, zaručoval, že $y > 0$.

% TODO: navazat treba na Frama-c, ze pokud dokazeme takovyto priklad, tak vsechny
% TODO: i nelogicke priklady po tom jsou podminene pravdive, nehlede na to, ze
% TODO: ze nemaji treba splnitelnost = assert \false