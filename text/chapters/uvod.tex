\chapter*{Úvod}
\addcontentsline{toc}{chapter}{Úvod}
\markboth{Úvod}{Úvod}

Formální verifikace softwaru je disciplína zabývající se dokazováním korektnosti programů
vzhledem k jejich specifikaci pomocí matematických metod a logických důkazů.
Na rozdíl od klasického testování, které se omezuje na ověřování konečného množství konkrétních vstupů,
formální verifikace usiluje o úplné pokrytí všech možných běhových stavů programu.
Využití formální verifikace je zásadní zejména v oblastech,
kde selhání softwaru může vést k závažným důsledkům.
Například v systémech řízení letového provozu,
medicínských přístrojích nebo například při verifikaci překladačů a kompilovaných jazyků.

Předmětem této diplomové práce je porovnání dvou nástrojů určených pro formální verifikaci programů,
které vycházejí z různých paradigmat programování a podporují různé programovací jazyky.
Prvním z nich je prostředí Frama\mbox{-}C pro jazyk C,
které umožňuje zápis formálních specifikací pomocí anotačního jazyka ACSL\@.
Druhým nástrojem je Stainless pro jazyk Scala.
Oba nástroje jsou založeny na podobných principech,
konkrétně Hoareově logice a metodě nejslabšího předpokladu.

\hyperref[ch:smt]{První kapitola} se věnuje SMT (Satisfiability Modulo Theories) řešičům,
principům jejich fungování a představení standardizovaného jazyka SMT-LIB\@.
\hyperref[ch:hoareova-logika]{Druhá kapitola} představuje Hoareovu logiku,
ze které vychází principy pro deduktivní verifikaci softwaru.
Na základě Hoareovy logiky je ve \hyperref[ch:metoda-nejslabsiho-predpokladu]{třetí kapitole}
představena metoda nejslabšího předpokladu,
která je jednou z nejběžnějších metod pro automatické generování důkazních postupů
a deduktivní verifikaci programů.
Kapitola~\ref{ch:frama-c} se věnuje nástroji Frama\mbox{-}C,
architektuře paměťových modelů a způsobu reprezentace paměti.
Obsahuje také přehled možných metod pro řešení
problematiky s důkazy na dynamických datových strukturách,
jako jsou například spojové seznamy nebo stromy.
Kapitola~\ref{ch:stainless} se zaměřuje na nástroj Stainless,
který je v mnoha aspektech podobný Frama\mbox{-}C,
ale specializuje se na verifikaci programů napsaných v jazyce Scala.
Stainless přistupuje k verifikaci dynamických datových struktur
odlišně a tato kapitola také obsahuje plnohodnotný důkaz
korektnosti implementace AVL stromu.

\newpage

Text této práce je dostupný také ve veřejném represitáři na GitHubu autora
\url{https://github.com/zapotocnylubos/masters-thesis-text}.
Ukázky kódu z společně se zdrojovým kódem oveřeného AVL stromu
jsou dostupné v repozitáři \url{https://github.com/zapotocnylubos/masters-thesis-code}.
