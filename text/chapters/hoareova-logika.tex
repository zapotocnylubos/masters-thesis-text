\chapter{Hoareova logika}
\label{ch:hoareova-logika}

Formální systém označovaný jako Hoareova logika byl navržen a popsán
britským matematikem a informatikem C. A. R. Hoarem v roce 1969~\cite{Hoare1969}.
Jedná se o formální systém pro popis a analýzu počítačových programů,
který pro důkaz správnosti programů používá předpoklady (preconditions) a následky (postconditions).

Program popisujeme pomocí Hoareovy trojice

\begin{equation*}
    \{P\} \  Q \  \{R\}
\end{equation*}

kde $P$ jsou předpoklady popisující stav systému před provedením programu,
$Q$ je daný program (příkaz či sekvence příkazů)
a $R$ jsou následky, které popisují stav systému po provedení tohoto programu.
Takto popsaný systém čteme následovně:
``Při splněných předpokladech $P$ budou po provedení programu $Q$ zaručeny následky $R$``.

Hoare v roce 1969 definoval trojici jako $P \ \{ Q \} \  R$,
ale v současnosti se častěji setkáme se zápisem $\{ P \} \  Q \ \{ R \}$.

Předpoklady a následky jsou vyjádřeny jako logické formule,
které popisují vlastnosti proměnných programu v určitém okamžiku.
Následující příklad reprezentuje program, který předpokládá na vstupu nezápornou hodnotu v proměnné $x$
a po provedení programu $Q$ bude zajištěno, že hodnota proměnné $x$ bude kladná (ostře větší než 0).

\begin{equation*}
    \{ x \geq 0 \} \  x \coloneqq x + 1 \  \{ x > 0 \}
\end{equation*}

Předpoklad může být také prázdný, což znamená, že program nevyžaduje žádné speciální podmínky pro spuštění
a formálně tuto situci lze zapsat jako $\{ true \} \  Q \  \{ R \}$.
Pokud je předpoklad prázdný, předpokládáme, že program může být spuštěn
v libovolném stavu systému nebo s libovolnými hodnotami proměnných.
Takový program může vypadat následovně:

\begin{equation*}
    \{ true \} \  x \coloneqq 10 \  \{ x > 0 \}
\end{equation*}

% TODO: zminit, ze toto je validni program, ale casto nam tento vysledek nic nerika a je potreba
% TODO: jeste pridat podminky na hodnoty promennych v ruznem case atd...

\section{Axiom přiřazení}
\label{sec:hoare-axiom-prirazeni}

Axiom přiřazení je základním pravidlem Hoareovy logiky a nedílnou součástí každého počítačového programu.
Přiřazení je operace obecně zapsaná ve tvaru

\begin{equation*}
    x \coloneqq f
\end{equation*}

kde $x$ je proměnná a $f$ je výraz bez vedlejších efektů (side effect),
který ale může obsahovat proměnnou $x$, například $x \coloneqq x + 1$.

Chceme zajistit, že jakékoli tvrzení $T$ platné o $f$
je také platné o hodnotě proměnné $x$ po provedení přiřazení.
Označíme si $T[x \leftarrow f]$ tvrzení $T$ ve kterém výskyt proměnné $x$ nahradíme výrazem $f$ (substituce).

Poté můžeme definovat axiom přiřazení jako

\begin{equation*}
    \{ T[x \leftarrow f] \} \  x \coloneqq f \  \{ T \}
\end{equation*}

Tento zápis říká, že pokud je pravdivé tvrzení $T$, ve kterém
je výskyt proměnné $x$ nahrazen výrazem $f$, pak po provedení přiřazení
je tvrzení $T$ pravdivé také pro proměnnou $x$.
Jedná se tedy o šablonu (schéma) axiomu,
kterou lze použít pro libovolné tvrzení $T$, proměnnou $x$ a výraz $f$.

% TODO: The assignment axiom proposed by Hoare does not apply when more than one name may refer to the same stored value. For example,
% TODO: end of https://en.wikipedia.org/wiki/Hoare_logic#Assignment_axiom_schema
% TODO: toto ma mozna konsekvence do Frama-C pri volne pametoveho modelu a proc
% TODO: nektere tvrzeni neprojdou na pointerech a rekurzivnich strukturach

\section{Pravidlo důsledku}
\label{sec:hoare-pravidlo-dusledku}

Pravidlo důsledku (rule of consequence) je metoda umožnující
tvorbu nových logických tvrzení z již existujících a dokázaných tvrzení.
Základní aplikací této metody je pravidlo \textbf{rozšíření předpokladu}
a pravidlo \textbf{zůžení následku}.
Předpokládejme platnosti $\{ P \} \  Q \  \{ R \}$.

Máme-li rozšířený předpoklad $P'$, pro který platí

\begin{equation*}
    P' \implies P
\end{equation*}

můžeme říci, že

\begin{equation*}
    \{ P' \} \  Q \  \{ R \}
\end{equation*}

je také platné tvrzení.

Máme-li zúžený následek $R'$ následku $R$, pro který platí

\begin{equation*}
    R \implies R'
\end{equation*}

můžeme říci, že $\{ P \} \  Q \  \{ R' \}$ je také platné tvrzení.

\section{Pravidlo skládání}
\label{sec:hoare-pravidlo-skladani}

Pravidlo skládání (rule of composition) je pravidlo, které umožňuje
skládat více příkazů do jednoho složeného příkazu a používat jeje v Hoareově logice.

Máme-li dva příkazy $Q_1$ a $Q_2$, které splňují následující tvrzení

\begin{equation*}
    \{ P_1 \} \  Q_1 \  \{ R_1 \}
\end{equation*}

a

\begin{equation*}
    \{ R_1 \} \  Q_2 \  \{ R_2 \}
\end{equation*}

můžeme říci, že složený příkaz $Q_1; Q_2$ splňuje následující tvrzení

\begin{equation*}
    \{ P_1 \} \  Q_1; Q_2 \  \{ R_2 \}
\end{equation*}

kde $;$ je operátor sekvence příkazů, který říká, že příkaz $Q_1$ bude proveden před příkazem $Q_2$.

Toto pravidlo umožňuje vytvářet složitější příkazy z několika jednodušších příkazů.
Zároveň je možné rozmyslet, že na sekvenci příkazů $Q_1, Q_2, \cdots, Q_n$
lze aplikovat stejné pravidlo skládání, které jsme použili pro dva příkazy
pomocí asociativity operátoru $;$ a závorek. A tedy platí, že

\begin{equation*}
    \{ P \} \  Q_1; Q_2; \  \cdots ; Q_n \  \{ R \}
\end{equation*}

je ekvivalentní s

\begin{equation*}
    \{ P \} \  (Q_1; (Q_2; \  \cdots (Q_{n-1}; Q_n))) \  \{ R \}
\end{equation*}

\section{Pravidlo iterace}
\label{sec:hoare-pravidlo-iterace}

Základním stavebním blokem počítačového programu je cyklus.
V Hoareově logice je cyklus reprezentován pomocí pravidla iterace (rule of iteration)
a využívá $while$ cyklus, který je v programovacích jazycích běžně dostupný a je definován následovně:

\begin{equation*}
    while \  B \  do \  Q
\end{equation*}

kde $Q$ je tělo cyklu, které se v každé iteraci provádí, dokud je podmínka $B$ pravdivá.

Dále definujeme invariant cyklu $I$, který nám pomůže
při rozhodování o správnost cyklu a bude popisovat následky provedení cyklu.
Invariant cyklu je logické tvrzení, které musí být pravdivé před vstupem do cyklu,
po ukončení každé iterace cyklu a také po ukončení cyklu.

Formálně můžeme zapsat podmínky pro invariant cyklu $I$ jako

\begin{equation*}
    I \land (\{ B \} \  Q \  \{ I \})
\end{equation*}

První část konjunkce popisuje, že invariant cyklu musí být pravdivý před vstupem do cyklu.
Druhá část konjunkce popisuje, že invariant cyklu musí být pravdivý po provedení těla cyklu $Q$,
pokud se cyklus spustil (podmínka $B$ byla pravdivá).

\begin{remark}
    Invariant cyklu je nezávislý na počtu provedených iterací.
\end{remark}

Pokud cyklus neprovedl žádnou iteraci a zároveň máme zaručeno,
že invariant cyklu $I$ byl pravdivý před vstupem do cyklu,
triviálně platí, že invariant cyklu $I$ je pravdivý i po ukončení cyklu.
Zároveň platí, že cyklus se neprovedl, protože podmínka $B$ byl nepravdivá,

Pokud cyklus provedl alespoň jednu iteraci a zároveň máme zaručeno,
že invariant cyklu $I$ je pravdivý po ukončení těla cyklu $Q$,
znamená to, že invariant cyklu $I$ je pravdivý i po ukončení poslední iterace cyklu.
Zároven platí, že podmínka $B$ je po ukončení cyklu nepravdivá, jinak by cyklus pokračoval v provádění další iterace.

Formálně lze tedy konstrukci $while$ cyklu pomocí Hoareovy logiky zapsat jako

\begin{equation*}
    I \land \{ B \} \  Q \  \{ I \} \implies \{ I \} \  while \  B \  do \  Q \  \{ \neg B \land I \}
\end{equation*}

% TODO: St´ale je vˇsak nutn´e dodat invarianty cykl˚u – jejich nalezen´ı je algoritmicky neˇreˇsiteln´e.
% TODO: je to pravda? (smt-bc-thesis.pdf)