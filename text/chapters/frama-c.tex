\chapter{Frama-C}
\label{ch:frama-c}

V oblasti formální verifikace softwaru hrají významnou roli nástroje umožňující
statickou analýzu zdrojového kódu a deduktivní dokazování správnosti těchto programů.
Mezi ustálené platformy zaměřené na analýzu programů napsaných v jazyce C patří prostředí Frama\mbox{-}C,
které poskytuje užitečné nástroje a možnost rozšíření o moduly a pluginy~\cite{FCKernelMaroneze2024}.
Frama\mbox{-}C je open-source nástroj, který byl vyvinut na výzkumném ústavu CEA-LIST (Commissariat à l'Énergie Atomique et aux Énergies Alternatives)
a první verze byla vydána v roce 2008.

Jádro Frama\mbox{-}C je napsáno v jazyce OCaml a je postaveno jako modulární framework
navržený s cílem usnadnit aplikaci pokročilých technik formální analýzy nad programy v jazyce C
pomocí integrace různých analytických nástrojů ve formě pluginů~\cite{FCPluginDevSignoles2024}.
Některé z těchto pluginů jsou obsaženy přímo v jádře Frama\mbox{-}C, zatímco další jsou dostupné jako externí moduly.
Moduly obsažené v jádře Frama\mbox{-}C zahrnují například deduktivní analyzátor WP (Weakest Precondition)
založený na teoretickém základu popsaném v kapitole~\ref{ch:metoda-nejslabsiho-predpokladu},
RTE (Run-Time Error) pro detekci chyb při běhu programu, který si představíme v kapitole~\ref{sec:frama-c-rte},
nebo například EVA (Evolving Value Analysis) pro analýzu hodnot proměnných v průběhu vykonávání programu.

Klíčovým prvkem Frama\mbox{-}C je definice a podpora specifikačního jazyka ACSL (ANSI/ISO C Specification Language),
který umožňuje uživatelům vyjadřovat vlastnosti a specifikace programů v jazyce C pomocí specifikačních komentářů.
Tyto komentáře anotují kód a poskytují informace pro statickou analýzu a deduktivní dokazování.
Návrh a implementace ACSL byly inspirovány podobným standardem pro jazyk Java, známým jako JML (Java Modeling Language)~\cite{ACSLSpec}.
ACLS bude podrobněji představen v kapitole~\ref{sec:acsl}.

Prostředí Frama\mbox{-}C je distribuováno jako program pro příkazový řádek a také jako grafické uživatelské rozhraní (GUI).
GUI poskytuje uživatelsky přívětivé prostředí pro interakci s celým ekosystémem Frama\mbox{-}C a hlavně s nainstalovanými pluginy.
Grafické prostředí je dostupné hlavně pro operační systémy Linux a Windows.
Pro uživatele operačního systému macOS je k dispozici pouze příkazová řádka.

Organizace stojící za vývojem Frama\mbox{-}C distribuuje mimo jiné také Docker image, které obsahují všechny potřebné závislosti
pro běh Frama\mbox{-}C a základních pluginů společně s několika SMT řešičemi, jako je Alt-Ergo, CVC4 a Z3.
Také distribuují GUI variantu imagů a je tedy velmi snadné spustit Frama\mbox{-}C GUI pomocí Dockeru na libovolném operačním systému,
jako je například macOS nebo Windows.
Tyto image obsahují minimalistické desktopové prostředí, Frama\mbox{-}C GUI a VNC aplikaci
umožnující připojení k desktopovému prostředí pomocí webového prohlížeče~\cite{FCDockerGUIMaroneze2021}.


\section{ACSL (ANSI/ISO C Specification Language)}
\label{sec:acsl}

ACLS je specifikační jazyk pro jazyk C, který umožňuje uživatelům
vyjadřovat vlastnosti a specifikace programů pomocí anotací v podobě speciálních komentářů ve zdrojovém kódu.

Anotace lze zapsat jako jednořádkový komentář \texttt{//@ ...} nebo víceřádkový komentář \texttt{/*@ ... */}.
Tyto anotace jsou umístěny přímo ve zdrojovém kódu a vypadají například jako v ukázce~\ref{list:acsl-example}.

\begin{listing}[H]
    \begin{minted}{C}
    /*@ requires x > 0;
        assigns \result;
        ensures \result == x + 2;
    */
    int increment(int x) {
        x = x + 1;
        //@ assert x > 1;
        return x + 1;
    }
    \end{minted}
    \caption{Ukázka anotace funkce v jazyce C pomocí ACSL}
    \label{list:acsl-example}
\end{listing}

V návaznosti na kapitolu~\ref{ch:metoda-nejslabsiho-predpokladu}, kde jsme se seznámili
s metodou nejslabšího předpokladu, si nyní ukážeme, jakým způsobem lze pomocí ACSL
vyjádřit vlastnosti, předpoklady a závěry pro funkce v jazyce C.

V ukázce~\ref{list:acsl-example} je uvedena funkce \texttt{increment}, která
definuje předpoklad a závěř pomocí klíčových slov \texttt{requires} a \texttt{ensures}.
Zjednodušená Hoaerova trojice pro příklad z kódu~\ref{list:acsl-example} vypadá následovně:

\begin{equation*}
    \{ x > 0 \} \ (x = x + 1; return \  x + 1) \ \{ result == x + 2 \}
\end{equation*}

Frama\mbox{-}C před spuštěním analýzy převede zdrojový kód do mezi-interpretace (intermediate representation)
nazývané \texttt{CIL} (C Intermediate Language)~\cite{Necula2002CIL}.
Frama\mbox{-}C používá vlastní verzi \texttt{CIL}, která je založena na původní verzi, kterou vytvořil George Necula.
Od roku 2016 tato původní verze již není udržována, ale Frama\mbox{-}C stále podporuje vlastní verzi.

Výhodou vlastního předzpracování a použití mezi-interpretace je,
že Frama\mbox{-}C může analyzovat zdrojový kód efektivně
a nainstalované pluginy mohou pracovat s touto abstraktní reprezentací
a nebo ji dokonce upravovat či transformovat~\cite{FCKernelMaroneze2024}.


\subsection{Anotace \texttt{\textbackslash at} a časové body}
\label{subsec:anotace-at-a-casove-body}

Anotaci \texttt{\textbackslash at(x, L)} můžeme využít pro jakékoliv místo v programu anotované jako $L$.
Návěští (label) $L$ může být například začátek funkce, začátek cyklu nebo libovolné místo v programu, které je ve zdrojovém kódu explicitně označeno.
Uživatelem definovaná návěští jsou standardní součástí jazyka C a Frama\mbox{-}C je používá pro specifikiaci časových bodů v programu.

Následující příklad~\ref{list:label-example} ukazuje, jakým způsobem lze definovat uživatelská návěští

Například v následujícím programu~\ref{list:label-example}:

\begin{listing}[H]
    \begin{minted}{C}
    void calc(int x) {
        L0:
        x++;
        L1:
        x += 2;
        //@ assert x >= \at(x, L0) + 3;
    }
    \end{minted}
    \caption{Ukázka uživatelského návěští v Frama-C}
    \label{list:label-example}
\end{listing}

jsou explicitně definovaná dvě návěští \texttt{L0} a \texttt{L1},
na která je možné se odkazovat pomocí anotace \texttt{\textbackslash at(x, L0)} a \texttt{\textbackslash at(x, L1)}.

Frama\mbox{-}C poskytuje implicitní návěští, která jsou automaticky generována,
konrkrétně jsou k dispozici tyto návěští~\cite{ACSLSpec}:

\begin{itemize}
    \item \textbf{Pre}
    \begin{itemize}
        \item Použitelné pouze v anotacích příkazů, nikoliv v kontraktech funkcí.
        \item Odkazuje na stav před začátkem funkce, ve které se anotace nachází.
    \end{itemize}

    \item \textbf{Post}
    \begin{itemize}
        \item Použitelné pouze v kontraktech funkcí v klauzulích \texttt{assigns} a \texttt{ensures}.
        \item Odkazuje na stav po ukončení funkce, ve které se anotace nachází.
    \end{itemize}

    \item \textbf{Here}
    \begin{itemize}
        \item Použitelné v anotacích příkazů i kontraktech funkcí.
        \item U klauzulí \texttt{requires}, \texttt{assumes}, \texttt{assigns}, \texttt{frees}, \texttt{decreases}, \texttt{terminates} odkazuje na \texttt{Pre} stav.
        \item U klauzulí \texttt{ensures}, \texttt{allocates} a při ukončení s výjimkou odkazuje na \texttt{Post} stav.
        \item V ostatních klauzulích odkazuje na aktualní stav v místě, kde se anotace nachází.
    \end{itemize}

    \item \textbf{LoopEntry}
    \begin{itemize}
        \item Použitelné v anotacích cyklů.
        \item Odkazuje na stav před provedením prvním iterace cyklu.
    \end{itemize}

    \item \textbf{LoopCurrent}
    \begin{itemize}
        \item Použitelné v anotacích cyklů.
        \item Odkazuje na stav před provedením aktuální iterace cyklu.
    \end{itemize}

    \item \textbf{Old}
    \begin{itemize}
        \item Použitelné v anotacích příkazů i kontraktech funkcí.
        \item U klauzulí \texttt{assigns} a \texttt{ensures} odkazuje na stav před provedením funkce.
        \item Oproti \texttt{Pre} lze \texttt{Old} použít i v kontraktech funkcí.
        \item ACSL definuje \texttt{\textbackslash old(x)} jako syntaktický cukr pro \texttt{\textbackslash at(x, Old)}.
    \end{itemize}
\end{itemize}

TODO: pointery a validita, separace

TODO: kvantifikátory, predikáty, logické funkce

TODO: loop invarianty, loop varianty

TODO: ghost konstrukty


\section{RTE (Run-Time Error)}
\label{sec:frama-c-rte}


\section{WP (Weakest Precondition)}
\label{sec:frama-c-wp}


\section{Paměťové modely}
\label{sec:frama-c-mem}

% TODO: objasnit co znamená ...chovají vzhledem k paměti...
Paměťové modely jsou důležitou součástí analýzy programů,
protože umožňují definovat, jakým způsobem se programy chovají
vzhledem k paměti a jakým způsobem manipulují s daty.

Paměťové modely popisují způsob popisu operace na

\subsection{Hoaerův paměťový model}
\label{subsec:hoaeruv-pametovy-model}

Nejjednodušší paměťový model je Hoareův paměťový model,
ve kterém se operace s pamětí provádějí pomocí přiřazení do logických proměnných.
Jedná se o aplikaci metody Single Static Assignment (SSA),
která zajišťuje, že každá logická proměnná má přiřazenou hodnotu pouze jednou.

Hoaerův paměťový model reprezentuje proměnné a jejich hodnoty v různých časech
pomocí několika různých logických proměnných, na které lze odkazovat pomocí
ACLS anotace \texttt{\textbackslash at}.
Tato anotace, její použití a význam byly podrobněji popsány v kapitole~\ref{subsec:anotace-at-a-casove-body}.

Následující příklad~\ref{list:ssa-example} popisuje velmi jednoduchou funkci,
která přijímá jeden parametr $x$ typu \texttt{int} (celé číslo) a provádí na něm dvě operace.
První operace zvyšuje hodnotu $x$ o 1 a druhá operace zvyšuje hodnotu $x$ o 2.
Provedením těchto operací se tedy hodnota proměnné $x$ zvětší o 3.
Pomocí anotace \texttt{\textbackslash at(x, Pre)} popisujeme počáteční hodnotu proměnné $x$,
tedy původní hodnotu parametru funkce.

\begin{listing}[H]
    \begin{minted}{C}
    void calc(int x) {
        x++;
        x += 2;
        //@ assert x >= \at(x, Pre) + 3;
    }
    \end{minted}
    \caption{Ukázka Single Static Assignment ve Frama-C}
    \label{list:ssa-example}
\end{listing}

Pomocí Frama\mbox{-}C lze nahlédnout na interní reprezentaci proměnných pomocí příkazu~\ref{list:ssa-example-run}.
Tento příkaz spustí analýzu zdrojového kódu v souboru \texttt{ssa.c} pomocí metody nejslabšího předpokladu.
Dále explicitně specifikujeme, že chceme použít paměťový model \texttt{Hoare},
jelikož Frama\mbox{-}C podporuje více paměťových modelů a výchozím modelem je \texttt{Typed},
který je popsán v následující kapitole~\ref{subsec:typovy-pametovy-model}.
Přepínač \texttt{-wp-no-let} zakazuje zjednodušování a propagaci rovností a substitucí,
bez kterého by výstup byl pouze označen za platný (valid) interním Qed zjednodušovačem (simplifier).
Qed je zjednodušovač, který se používá před odesláním dotazů do SMT solverů,
primárně z důvodu zjednodušení dotazů a zrychlení analýzy~\cite{WPManual}.
Přepínač \texttt{-wp-print} specifikuje, že chceme zobrazit detaily analýzy a vypsat je na standardní výstup.

\begin{listing}[H]
    \begin{minted}{console}
    frama-c -wp -wp-model="Hoare" -wp-no-let -wp-print ssa.c
    \end{minted}
    \caption{Příkaz pro zobrazení SSA ve Frama-C}
    \label{list:ssa-example-run}
\end{listing}

Výstup analýzy~\ref{list:ssa-frama-c} potvrzuje, že při každém přiřazení do proměnné $x$ vzniká nová logická proměnná.
Logické proměnné jsou číslované ve vzestupném pořadí a číslují se v obráceném pořadí, než v jakém byly přiřazeny.
Tedy poslední přiřazení do proměnné $x$ je reprezentováno proměnnou $x_1$, předposlední přiřazení $x_2$ a tak dále.
Samostatná logická proměnná $x$ reprezentuje hodnotu proměnné $x$ před prvním přiřazením,
tedy na začátku funkce a jedná se o hodnotu proměnné $x$, kterou získáváme voláním \texttt{\textbackslash at(x, Pre)}.

\begin{listing}[H]
    \begin{minted}{text}
    Goal Assertion (file ssa.c, line 4):
    Assume {
      Type: is_sint32(x_1) /\ is_sint32(x)
            /\ is_sint32(x_2) /\ is_sint32(x_3).
      Have: x_3 = x.
      Have: (1 + x_3) = x_2.
      Have: (2 + x_2) = x_1.
    }
    Prove: (3 + x) <= x_1.
    Prover Qed returns Valid
    \end{minted}
    \caption{Ukázka SSA ve Frama-C}
    \label{list:ssa-frama-c}
\end{listing}

% TODO: sem sepsat nevyhody Hoaerova pametoveho modelu

Hoareův paměťový model a SSA jsou velmi užitečné metody pro analýzu programů,
ale mají své omezení.
Největší nevýhodou je, že není je

\subsection{Typový paměťový model}
\label{subsec:typovy-pametovy-model}

\subsection{Referenční paměťový model}
\label{subsec:referencni-pametovy-model}

% TODO: jo?
% -wp-model <model+...>  Memory model selection. Available selectors:
% * 'Hoare' logic variables only
% * 'Typed' typed pointers only
% * '+nocast' no pointer cast
% * '+cast' unsafe pointer casts
% * '+raw' no logic variable
% * '+ref' by-reference-style pointers detection
% * '+nat/+int' natural / machine-integers arithmetics
% * '+real/+float' real / IEEE floating point arithmetics
% * 'Eva' (experimental) based on the results from Eva
% plugin (preferably use -wp-model="model+...")
% ====
%\subsection{Problémy paměťových modelů}
