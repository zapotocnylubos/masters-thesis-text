\chapter{Frama-C}
\label{ch:frama-c}

V oblasti formální verifikace softwaru hrají významnou roli nástroje umožňující
statickou analýzu zdrojového kódu a deduktivní dokazování správnosti těchto programů.
Mezi ustálené platformy zaměřené na analýzu programů napsaných v jazyce C patří prostředí Frama-C,
které poskytuje užitečné nástroje a možnost rozšíření o moduly a pluginy~\cite{FCKernelMaroneze2024}.
Frama-C je open-source nástroj, který byl vyvinut na výzkumném ústavu CEA-LIST (Commissariat à l'Énergie Atomique et aux Énergies Alternatives)
a první verze byla vydána v roce 2008.

Jádro Frama-C je napsáno v jazyce OCaml a je postaveno jako modulární framework
navržený s cílem usnadnit aplikaci pokročilých technik formální analýzy nad programy v jazyce C
pomocí integrace různých analytických nástrojů ve formě pluginů~\cite{FCPluginDevSignoles2024}.
Některé z těchto pluginů jsou obsaženy přímo v jádře Frama-C, zatímco další jsou dostupné jako externí moduly.
Moduly obsažené v jádře Frama-C zahrnují například deduktivní analyzátor WP (Weakest Precondition)
založený na teoretickém základu popsaném v kapitole~\ref{ch:metoda-nejslabsiho-predpokladu},
RTE (Run-Time Error) pro detekci chyb při běhu programu, který si představíme v kapitole~\ref{sec:frama-c-rte},
nebo například EVA (Evolving Value Analysis) pro analýzu hodnot proměnných v průběhu vykonávání programu.

Klíčovým prvkem Frama-C je definice a podpora specifikačního jazyka ACSL (ANSI/ISO C Specification Language),
který umožňuje uživatelům vyjadřovat vlastnosti a specifikace programů v jazyce C pomocí specifikačních komentářů.
Tyto komentáře anotují kód a poskytují informace pro statickou analýzu a deduktivní dokazování.
Návrh a implementace ACSL byly inspirovány podobným standardem pro jazyk Java, známým jako JML (Java Modeling Language)~\cite{ACSLSpec}.
ACLS bude podrobněji představen v kapitole~\ref{sec:acsl}.

\section{ACSL (ANSI/ISO C Specification Language)}
\label{sec:acsl}


\section{RTE (Run-Time Error)}
\label{sec:frama-c-rte}


\section{WP (Weakest Precondition)}
\label{sec:frama-c-wp}


\section{Paměťové modely}
\label{sec:frama-c-mem}

\subsection{Hoaerův paměťový model}

\subsection{Typový paměťový model}

\subsection{Referenční paměťový model}
