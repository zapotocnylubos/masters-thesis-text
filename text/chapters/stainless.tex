\chapter{Stainless}
\label{ch:stainless}

Stainless je nástroj pro ověřování korektnosti programů napsaných v jazyce Scala.
Scala je staticky typovaný jazyk, který je založen na jazyce Java.
Scala podporuje objektově orientované programování i funkcionální styl programování.
Nicméně podporovaný je hlavně funkcionální styl programování a některé konstrukce
imperativního programování jsou podporovány pomocí transformací, které jsou součástí
předzpracování zdrojového kódu~\cite{StainlessDocs}.

Stainless stejně jako Frama\mbox{-}C používá SMT řešiče pro dokazování vlastností programů.
Místo platformy Why3 ale používá vlastní mezivrstvu nazvanou Inox.
Inox podporuje generování důkazů pro externí SMT řešiče Z3, CVC4 a CVC5.
Interní řešič Princess, napsaný v jazyce Scala, je součástí distribuce a je ho tedy možné použít
i v případě, kdy není k dispozici žádný externí SMT řešič~\cite{InoxSolver}.

\section{Anotace}
\label{sec:stainless-annotations}

Stainless podporuje podobný styl anotování kódu jako Frama\mbox{-}C\@.
Není ale potřeba používát speciální komentáře pro anotace,
ale je možné použít přímo jazyk Scala.
Všechny anotace jsou umístěny jako spustitelný Scala kód a nebo jako dekorátory před funkcemi.

Anotace \texttt{require} zajišťuje, že daná podmínka musí být splněna před spuštěním funkce.
Příklad použití je zobrazen v ukázce~\ref{lst:stainless-require},
kde je deklarována funkce \texttt{my\_function} s dvěma celočíselnými parametry.
Podmínka spuštění této funkce je, že oba parametry musí být ostře větší než 0.

\begin{listing}[H]
  \begin{minted}{scala}
  def my_function(x: Int, y: Int): Int = {
    require(x > 0 && y > 0)
    ...
  }
  \end{minted}
  \caption{Příklad použití anotace \texttt{require}}
  \label{lst:stainless-require}
\end{listing}

Anotace \texttt{ensuring} zajišťuje, že po spuštění funkce budou splněné dané vlastnosti.
Ensuring přijímá jako argument funkci o jednom parametru,
která vrací hodnotu typu \texttt{Boolean}.
Parametrem této funkce je výstupní hodnota funkce,
která je na ukázce~\ref{lst:stainless-ensuring} pojmenována \texttt{res}.

\begin{listing}[H]
  \begin{minted}{scala}
  def my_function(x: Int, y: Int): Int = {
    ...
  }.ensuring(res => res > 0)
  \end{minted}
  \caption{Příklad použití anotace \texttt{ensuring}}
  \label{lst:stainless-ensuring}
\end{listing}

\subsection{Cykly}
\label{subsec:stainless-loops}

Cykly v jazyce Scala jsou jedním z imperativních prvků,
které Stainless musí transformovat na funkcionální styl.
Stejně jako při použití Frama\mbox{-}C je potřeba použít invarianty cyklu,
popisující vlastnosti cyklu.
Také je potřeba definovat variant cyklu,
pomocí kterého lze rozhodnout o konečnost cyklu.
Stainless posktytuje možnost definovat invariant pomocí klíčového slova \texttt{invariant}
umístěného za deklarací cyklu.
Variant je označen pomocí klíčového slova \texttt{decreases} na začátku těla cyklu.
Ukázka~\ref{lst:stainless-loop} zobrazuje jednoduchý příklad cyklu s použitím invariantu i variantu.
Při zápisu invariantu je nutné cyklus obalit do závorek,
jelikož Stainless používá pro integraci tohoto klíčového slova
implicitní konverzi z typu \texttt{Unit} na typ \texttt{InvariantFunction}~\cite{StainlessDocs}.

\begin{listing}[H]
  \begin{minted}{scala}
  def my_loop(x: Int): Int = {
    var i = x

    (while (i > 0) {
      decreases(i)

      i -= 1
    }).invariant(i >= 0)

    i
  }
  \end{minted}
  \caption{Příklad použití cyklu s invariantem a variantem}
  \label{lst:stainless-loop}
\end{listing}

\subsection{Pattern matching}
\label{subsec:stainless-pattern-matching}

Scala podporuje takzvaný pattern matching,
jehož funkce je podobná konstrukci \texttt{switch} v jazyce~C s několika vylepšeními.
Pomocí tohoto mechanismu je možné elegantně zpracovávat různé varianty datových typů.
Dokonce je možné rozpoznávat i složené datové typy pomocí destrukturizace.
Stainless také podporuje pattern matching a v následující kapitole s jeho pomocí
popíšeme implementaci AVL stromu~\cite{scalaPatternMatching}.

Scala podporuje koncept algebrických datových typů
a takzvaných \textit{case classes}.
Příkladem může být následující ukázka~\ref{lst:stainless-case-class},
kde je definována abstraktní třída \texttt{AVLTree} a dvě podtřídy \texttt{Node} a \texttt{Leaf},
které budou detailně popsány v kapitole~\ref{sec:stainless-avl}.
Přidáním příznaku \texttt{sealed} před definici třídy
je možné zajistit, že všechny podtřídy budou definované v rámci stejného souboru.

Kombinací pattern matchingu a \texttt{case class} je možné elegantně zpracovávat různé varianty datových typů.
Například pomocí kódu z ukázky~\ref{lst:stainless-pattern-matching-case-class}
je možné definovat funkci \texttt{content},
která pomocí pattern matchingu vypočítá množinu prvků obsažených v AVL stromu.

\begin{listing}[H]
  \begin{minted}{scala}
  sealed abstract class AVLTree {
    def content: Set[Int] = this match {
      case Leaf() => Set.empty
      case Node(v, l, r, _) => l.content ++ Set(v) ++ r.content
    }
  }
  \end{minted}
  \caption{Příklad použití pattern matchingu s \texttt{case class}}
  \label{lst:stainless-pattern-matching-case-class}
\end{listing}

Pattern matchingu bez Stainless může v některých případech vyvolat
varování o neúplnosti pattern matchingu, pokud \texttt{match} nezachycuje všechny \texttt{case class}
a neobsahuje fallback variantu pomocí konstrukce \texttt{case \_},
Toto varování Stainless vylepšuje a přidává kontext o spouštění funkce.
Pokud tedy v aktuálním kontextu některé případy nemohou nikdy nastat,
Stainless potlačí varování o neúplnosti pattern matchingu.

\section{Protipříklady}
\label{sec:stainless-counterexamples}

Velkou výhodou oproti Frama\mbox{-}C při vývoji důkazů pomocí Stainless je,
že Stainless v základní instalaci podporuje generování protipříkladů.
Protipříklad je konkrétní vstupní hodnota nebo nastavení proměnných,
které vedou k porušení kontraktu funkce.

Následující ukázka~\ref{lst:stainless-incorrect-intmax} zobrazuje
jednoduchou funkci \texttt{int\_max}, která vrací větší z dvou čísel.
Tato funkce je záměrně implementovana špatně,
aby bylo možné ukázat generování protipříkladů.
Definice kontraktu \texttt{ensuring} zajišťuje,
že výsledek volání funkce bude hodnota větší nebo rovna oběma vstupním parametrům.
Žádné vstupní podmínky (předpoklady) nejsou definované,
Stainless tedy předpokládá, že vstupní hodnoty jsou libovolné celočíselné hodnoty.
Pomocí příkazu~\ref{lst:stainless-incorrect-intmax-run}
lze spustit analýzu tohoto programu.

\begin{listing}[H]
  \begin{minted}{bash}
  def int_max(x: Int, y: Int): Int = {
    if (x > y) x else 0
  }.ensuring(res => res >= x && res >= y)
  \end{minted}
  \caption{Nesprávně implementovaná funkce \texttt{int\_max}}
  \label{lst:stainless-incorrect-intmax}
\end{listing}

\begin{listing}[H]
  \begin{minted}{console}
  stainless snippets/counterexample.scala
  \end{minted}
  \caption{Příkaz pro spuštění analýzy protipříkladů}
  \label{lst:stainless-incorrect-intmax-run}
\end{listing}

Výsledek analýzy z výpisu~\ref{lst:stainless-incorrect-intmax-result} ukazuje,
že pro vstupní hodnoty $x = -2147483648$ a $y = 2147483646$
je výstupní hodnota $0$, což je v rozporu s kontraktem funkce.
Stainless tyto protipříklady generuje automaticky,
ale již neprovádí jejich minimalizaci.

\begin{listing}[H]
  \begin{minted}{console}
  ...
  [Warning ] snippets/counterexample.scala:6:23:  => INVALID
               if (x > y) x else 0
                                 ^
  [Warning ] Found counter-example:
  [Warning ]   x: Int -> -2147483648
  [Warning ]   y: Int -> 2147483646
  ...
  \end{minted}
  \caption{Výstup analýzy s protipříkladem pro funkci \texttt{int\_max}}
  \label{lst:stainless-incorrect-intmax-result}
\end{listing}

Velmi užitečná je podpora protipříkladů i pro složitější datové struktury
jako například algebrické datové typy.
Například pro AVL strom, který bude popsán později v této kapitole,
byl v průběhu tvorby důkazu vygenerován protipříklad podobný jako z výpisu~\ref{lst:stainless-avl-counterexample},
ve kterém Stainless přesně popisuje protipříklad s kombinací datových typů \texttt{Node} a \texttt{Leaf}.
Zároveň je správně dodržena podmínka na výšku uzlu, která musí být ostře větší než 0.

\begin{listing}[H]
  \begin{minted}{console}
  [Warning ] Found counter-example:
  [Warning ]   thiss: AVLTree -> Node(0, Leaf(), Leaf(), 1)
  \end{minted}
  \caption{Výstup analýzy s protipříkladem pro AVL strom}
  \label{lst:stainless-avl-counterexample}
\end{listing}

% https://epfl-lara.github.io/stainless/genc.html
% The support for classes is restricted to non-recursive ones so that instances of such data-types live on the stack.
% ---
% Currently the memory model is limited to stack allocation and global state.
% Hence, no dynamic allocation is done using malloc function family.

% --genc does not support recursive data types = src/stainless/LinkedList.scala:41:12: Cons and other recursive types are not supported

\section{Ověřená implementace AVL stromu}
\label{sec:stainless-avl}

Datová struktura AVL stromu je rozšířením binárního vyhledávacího stromu,
který ale zajišťuje, že rozdíl výšek levého a pravého podstromu je maximálně 1.
Jedná se tedy o relaxovanou podmínku pro vyváženost stromu.
Tato datová struktura je velmi efektivní pro vyhledávání, vkládání a mazání prvků,
které mají pouze logaritickou složitost~\cite{Pruvodce22}.

Následující výpis~\ref{lst:stainless-avl-interface} zobrazuje rozhraní AVL stromu,
které je implementováno v jazyce Scala a následně ověřeno pomocí Stainless.
Datová struktura obsahuje základní operace pro vkládání, mazání a hledání prvku.
Jednotlivé funkce jsou korektně implementované a pomocí Stainless je možné ověřit jejich korektnost.
Třída \texttt{AVLTree} je abstraktní třída, která definuje základní vlastnosti AVL stromu
a podtřídy \texttt{Node} a \texttt{Leaf} implementují konkrétní chování stromu.
Zároveň je v kódu často používán pattern matching,
který umožňuje elegantně zpracovávat různé varianty situací, které mohou nastat.
List (Leaf) je prázdný uzel, který nemá žádné hodnoty a má výšku 0.
Jedná se o prázdný prvek, podobně jako by v jazyce C byl použit ukazatel \texttt{NULL}.
Nejedná se tedy o poslední platný uzel stromu, ale o zvláštní prvek, který nemá žádnou hodnotu.

\begin{listing}[H]
    \begin{minted}{scala}
    import stainless.lang._
    import stainless.collection._
    import stainless.annotation._

    def int_max(x: Int, y: Int): Int = ...

    sealed abstract class AVLTree {
      def content: Set[Int] = ...
      def height: Int = ...
      def balanceFactor: Int = ...
      def hasBinarySearchTreeStructure: Boolean = ...
      def hasAVLTreeStructure: Boolean = ...
      def isBalanced: Boolean = ...
      def isAVLTree: Boolean = ...
      def contains(x: Int): Boolean = ...
      def insert(x: Int): AVLTree = ...
      def delete(x: Int): AVLTree = ...
    }

    case class Leaf() extends AVLTree

    case class Node
    (
      value: Int,
      left: AVLTree,
      right: AVLTree,
      _height: Int
    ) extends AVLTree {
      require(_height > 0)

      def min: Int = ...
      def max: Int = ...
      def rotateLeft: Node = ...
      def rotateRight: Node = ...
      def rotateLeftRight: Node = ...
      def rotateRightLeft: Node = ...
      def balance: Node = ...
    }
    \end{minted}
    \caption{Rozhraní AVL stromu}
    \label{lst:stainless-avl-interface}
\end{listing}

Tato kapitola postupně popíše jednotlivé funkce a popíše důkazy jejich korektnosti.

Na začátku kódu je jednoduchá funkce \texttt{int\_max}, která vrací větší z dvou čísel.
Jedná se o pomocnou funkci na celých číslech, která se používá v dalších funkcích.
Stainless nepodporuje funkci \texttt{x max y} ze standardní knihovny,
proto je potřeba použít vlastní implementaci.

Ukázka~\ref{lst:stainless-avl-helper} zobrazuje pomocné funkce pro AVL strom.
První z nich je \texttt{content}, která vrací množinu prvků v AVL stromu.
Pomocí pattern matchingu se zjistí, zdali aktuálně zpracovávaný uzel je list nebo uzel.
Pokud je to list, vrátí se prázdná množina.
V případě uzlu se rekurzivně zavolá funkce \texttt{content} na levém a pravém podstromu
a spojí se dohromady do množiny s hodnotou uloženou v aktuálním uzlu.
Druhá funkce \texttt{height} vrací výšku stromu.
Listy stromu mají výšku 0 a uzly mají výšku uloženou v proměnné \texttt{\_height} (čtvrtý argument konstruktoru).
Třetí funkce \texttt{balanceFactor} vrací vyváženost stromu.
Listy jsou vždy vyvážené a uzly mají vyváženost počítanou jako rozdíl výšky pravého a levého podstromu.
V důkazech je použita tato vlastnost pro kontrolu vyváženosti stromu.

\begin{listing}[H]
  \begin{minted}{scala}
  def content: Set[Int] = this match {
    case Leaf() => Set.empty
    case Node(v, l, r, _) => l.content ++ Set(v) ++ r.content
  }

  def height: Int = this match {
    case Leaf() => 0
    case Node(_, _, _, h) => h
  }

  def balanceFactor: Int = {
    this match {
      case Leaf() => 0
      case Node(_, l, r, _) => r.height - l.height
    }
  }
  \end{minted}
  \caption{Pomocné funkce pro AVL strom}
  \label{lst:stainless-avl-helper}
\end{listing}

Jelikož je AVL strom nadstavbou binárního vyhledávacího stromu,
je potřeba zajistit, že splňuje všechny vlastnosti binárního vyhledávacího stromu.
K tomu slouží funkce \texttt{hasBinarySearchTreeStructure} zobrazená ve výpisu~\ref{lst:stainless-avl-bst},
která kontroluje hodnoty v uzlech a podmínky pro binární vyhledávací stromy.
Konkrétně, že levý podstrom obsahuje pouze hodnoty menší než hodnota aktuálního uzlu
a pravý podstrom pouze hodnoty větší než hodnota aktuálního uzlu.
Listy jsou automaticky označeny jako binární vyhledávací stromy (prázdný strom).
U uzlů se rekurzivně je nutné rekurzivně ověřit platnost pravidel pro
binární vyhledávací stromy na levém a pravém podstromu
a poté kontrolovat pravidlo pro binární vyhledávací stromy.

\begin{listing}[H]
  \begin{minted}{scala}
  def hasBinarySearchTreeStructure: Boolean = this match {
    case Leaf() => true
    case Node(v, l, r, _) => {
      l.hasBinarySearchTreeStructure &&
        r.hasBinarySearchTreeStructure &&
        // Ensure that the AVL
        //   has a binary search tree structure
        forall((x: Int) =>
          l.content.contains(x) ==> x < v
        ) &&
        forall((x: Int) =>
          r.content.contains(x) ==> x > v
        )
    }
  }
  \end{minted}
  \caption{Funkce pro kontrolu struktury binárního vyhledávacího stromu}
  \label{lst:stainless-avl-bst}
\end{listing}

Funkce \texttt{hasAVLTreeStructure} z výpisu~\ref{lst:stainless-avl-avl}
popisuje strukturu AVL stromu.
Tato funkce kontroluje, že prvky v AVL stromu jsou správně uspořádány.
Konkrétně je omezena hloubka stromu na maximálně \texttt{Int.MaxValue}
a výška stromu musí být správně vypočítaná (o jedna větší než maximální výška z podstromů).

\begin{listing}[H]
  \begin{minted}{scala}
  def hasAVLTreeStructure: Boolean = this match {
    case Leaf() => true
    case Node(v, l, r, _) => {
      l.hasAVLTreeStructure &&
        r.hasAVLTreeStructure &&
        // Ensure that the height of the tree
        //  is at most (at root) is Int.MaxValue
        l.height < Int.MaxValue &&
        r.height < Int.MaxValue &&
        // Ensure that the height of the tree is correct
        height == 1 + (l.height < r.height match {
          case true => r.height
          case false => l.height
        })
    }
  }
  \end{minted}
  \caption{Funkce pro kontrolu struktury AVL stromu}
  \label{lst:stainless-avl-avl}
\end{listing}

Funkce \texttt{isBalanced} z výpisu~\ref{lst:stainless-avl-balanced}
ověřuje, zdali je AVL strom vyvážený.
Kontroluje, zdali je rozdíl výšek pravého a levého podstromu
v rozsahu $[-1, 1]$.
Tato vývaženost musí rekurzivně platit pro všechny uzly v AVL stromu.

\begin{listing}[H]
  \begin{minted}{scala}
  def isBalanced: Boolean = this match {
    case Leaf() => true
    case Node(v, l, r, _) => {
      l.isBalanced &&
        r.isBalanced &&
        // Ensure that the balance factor
        //   is within [-1, 1]
        balanceFactor >= -1 &&
        balanceFactor <= 1
    }
  }
  \end{minted}
  \caption{Funkce pro ověření vyváženosti AVL stromu}
  \label{lst:stainless-avl-balanced}
\end{listing}

Funkce \texttt{isAVLTree} z výpisu~\ref{lst:stainless-avl-isavl}
ověřuje, že AVL strom splňuje všechny výše uvedené vlastnosti dohromady.

\begin{listing}[H]
  \begin{minted}{scala}
  def isAVLTree: Boolean = this match {
    case Leaf() => true
    case Node(v, l, r, h) => {
      l.isAVLTree &&
        r.isAVLTree &&
        hasBinarySearchTreeStructure &&
        hasAVLTreeStructure &&
        isBalanced
    }
  }
  \end{minted}
  \caption{Funkce pro ověření struktury AVL stromu}
  \label{lst:stainless-avl-isavl}
\end{listing}

Funkce \texttt{contains} z výpisu~\ref{lst:stainless-avl-contains}
ověřuje, zdali je daný prvek součástí AVL stromu.
Předpokladem spuštění této funkce je, že je spuštěna pouze na platném AVL stromu.
Následek popisující výsledek funkce zajišťuje,
že výsledek spuštění této funkce je stejný jako kontrola obsahu stromu pomocí množinové operace (\texttt{content.contains(x)}).

\begin{listing}[H]
  \begin{minted}{scala}
  def contains(x: Int): Boolean = {
    require(isAVLTree)
    this match {
      case Leaf() => false
      case Node(v, l, r, _) => {
        if (x == v) true
        else if (x < v) l.contains(x)
        else r.contains(x)
      }
    }
  }.ensuring(res => res == content.contains(x))
  \end{minted}
  \caption{Funkce pro kontrolu přítomnosti prvku v AVL stromu}
  \label{lst:stainless-avl-contains}
\end{listing}

Funkce \texttt{insert} z výpisu~\ref{lst:stainless-avl-insert}
provádí vložení prvku do AVL stromu.
V případě prázdného stromu se vytvoří nový uzel s danou hodnotou a výškou 1.
Pokud je strom již naplněn,
je potřeba zjistit, zdali prvek již ve stromu není.
Pokud není, je potřeba ho vložit na správné místo a případně provést opravu vyváženosti pomocí rotací.
Funkce \texttt{balance} bude představena později.
Prozatím stačí říci, že se jedná o funkci,
která dokáže opravit AVL strom i v případě,
kdy je balanční faktor aktuálního vrcholu v rozsahu $[-2, 2]$.

\begin{listing}[H]
  \begin{minted}{scala}
  def insert(x: Int): AVLTree = {
    require(isAVLTree)
    require(height < Int.MaxValue)

    this match {
      case Leaf() => Node(x, Leaf(), Leaf(), 1)
      case Node(v, l, r, _) => {
        if (x == v) this
        else if (x < v) {
          val newLeft = l.insert(x)
          Node(
            v, newLeft, r,
            1 + int_max(newLeft.height, r.height)
          ).balance
        } else {
          val newRight = r.insert(x)
          Node(
            v, l, newRight,
            1 + int_max(l.height, newRight.height)
          ).balance
        }
      }
    }
  }.ensuring(res =>
    res.isAVLTree &&
      (res.height == height || res.height == height + 1) &&
      res.content == content ++ Set(x)
  )
  \end{minted}
  \caption{Funkce pro vložení prvku do AVL stromu}
  \label{lst:stainless-avl-insert}
\end{listing}

Funkce \texttt{delete} z výpisu~\ref{lst:stainless-avl-delete}
provádí odstranění prvku z AVL stromu.
Jedná se o nejkomplexnejší z představených funkcí.
Pokud je aktuálně zpracovávaný uzel hledaný prvek na odstranění (\texttt{x == v}),
je potřeba rozlišit čtyři případy dle podstromů daného uzlu.
Pokud jsou oba podstromy prázdné, vrátí se instance listu (prázdný strom).
Vrácení prázdného listu je vlastně nahrazení daného vrcholu prázdným stromem.
Pokud pravý nebo levý podstrom zaplněný (ale ne oba),
vrátí se ten podstrom, který není prázdný.
Pokud jsou oba podstromy zaplněné,
je potřeba najít minimální prvek v pravém podstromu (následník),
odstranit ho z pravého podstromu
a vytvořit nový uzel s hodnotou následníka, stejným levým podstromem
a novým pravým podstromem, kde byl následník odstraněn.
V případech kdy je hledaný prvek menší nebo větší než hodnota aktuálního uzlu
je potřeba rekurzivně zavolat funkci \texttt{delete} na levém nebo pravém podstromu
a po provedení odstranění nebo ukončení v případě, kdy prvek není ve stromu,
je potřeba vytvořit nový uzel s novým levým nebo pravým podstromem
a provést opravu vyváženosti pomocí funkce \texttt{balance}.
Následky této funkce zajišťují, že se vždy vrátí korektní AVL strom
a že výška stromu se buď nezmění nebo se sníží maximálně o 1.
Zároveň je zajištěno, že obsah stromu je stejný jako v původním stromu,
s výjimkou odstraněného prvku.

\begin{listing}[H]
  \begin{minted}{scala}
  def delete(x: Int): AVLTree = {
    require(isAVLTree)
    this match {
      case Leaf() => this
      case Node(v, l, r, _) => {
        if (x == v) {
          (l, r) match {
            case (Leaf(), Leaf()) => Leaf()
            case (Leaf(), Node(_, _, _, _)) => r
            case (Node(_, _, _, _), Leaf()) => l
            case (Node(_, _, _, _), rn: Node) => {
              val m = rn.min
              val newRight = r.delete(m)
              Node(
                m, l, newRight,
                1 + int_max(l.height, newRight.height)
              ).balance
            }
          }
        } else if (x < v) {
          val newLeft = l.delete(x)
          Node(
            v, newLeft, r,
            1 + int_max(newLeft.height, r.height)
          ).balance
        } else {
          val newRight = r.delete(x)
          Node(
            v, l, newRight,
            1 + int_max(l.height, newRight.height)
          ).balance
        }
      }
    }
  }.ensuring(res =>
    res.isAVLTree &&
      (res.height == height || res.height == height - 1) &&
      res.content == content -- Set(x)
  )
  \end{minted}
  \caption{Funkce pro odstranění prvku z AVL stromu}
  \label{lst:stainless-avl-delete}
\end{listing}

Předchozí funkce \texttt{delete} využívala funkci \texttt{min} z výpisu~\ref{lst:stainless-avl-min-max},
která vrací minimální prvek v AVL stromu.
Pro kompletnost je na uzlech definována i funkce \texttt{max},
která vrací maximální prvek v AVL stromu.

\begin{listing}[H]
  \begin{minted}{scala}
  def min: Int = {
    require(isAVLTree)
    this match {
      case Node(v, Leaf(), _, _) => v
      case Node(_, l: Node, _, _) => l.min
    }
  }.ensuring(res =>
    content.contains(res) &&
      forall((x: Int) =>
        (content.contains(x) && x != res) ==> x > res
      )
  )

  def max: Int = {
    require(isAVLTree)
    this match {
      case Node(v, _, Leaf(), _) => v
      case Node(_, _, r: Node, _) => r.max
    }
  }.ensuring(res =>
    contains(res) &&
      forall((x: Int) =>
        (contains(x) && x != res) ==> x < res
      )
  )
  \end{minted}
  \caption{Funkce pro zjištění minimálního a maximálního prvku v AVL stromu}
  \label{lst:stainless-avl-min-max}
\end{listing}

Funkce \texttt{rotateLeft} a \texttt{rotateRight}
z výpisu~\ref{lst:stainless-avl-rotate}
provádějí základní rotace uzlů v AVL stromu.
Tyto funkce jsou volány v případě, kdy je potřeba opravit vyváženost stromu
a zajišťují, že strom zůstane AVL stromem a jeho obsah se nezmění.
Vstupní podmínka těchto funkcí nemůže být pouze \texttt{isAVLTree},
jelikož \texttt{isAVLTree} zajišťuje, že strom je vyvážený,
což na místech v programu, kde se tyto funkce volají není zaručeno.
Následky spuštění těchto funkcí zaručují,
že strom bude správně vyvážen a obsah zůstane nezměněn.

\begin{listing}[H]
  \begin{minted}{scala}
  def rotateLeft: Node = {
    require(hasBinarySearchTreeStructure)
    require(hasAVLTreeStructure)
    require(left.isAVLTree && right.isAVLTree)
    require(balanceFactor == 2 && right.balanceFactor >= 0)

    right match {
      case Node(v, rl, rr, _) =>
        Node(
          v,
          Node(
            value, left, rl,
            1 + int_max(left.height, rl.height)
          ),
          rr,
          1 + int_max(
            1 + int_max(left.height, rl.height),
            rr.height
          )
        )
    }
  }.ensuring(res => res.content == content && res.isAVLTree)

  def rotateRight: Node = {
    require(hasBinarySearchTreeStructure)
    require(hasAVLTreeStructure)
    require(left.isAVLTree && right.isAVLTree)
    require(balanceFactor == -2 && left.balanceFactor <= 0)

    left match {
      case Node(v, ll, lr, _) =>
        Node(
          v,
          ll,
          Node(
            value, lr, right,
            1 + int_max(lr.height, right.height)
          ),
          1 + int_max(
            1 + int_max(lr.height, right.height),
            ll.height
          )
        )
    }
  }.ensuring(res => res.content == content && res.isAVLTree)
  \end{minted}
  \caption{Funkce pro jednoduché rotace uzlů v AVL stromu}
  \label{lst:stainless-avl-rotate}
\end{listing}

Dvojité rotace \texttt{rotateLeftRight} a \texttt{rotateRightLeft}
z výpisu~\ref{lst:stainless-avl-rotateLeftRight} a~\ref{lst:stainless-avl-rotateRightLeft}
reprezentují levo-pravou a pravo-levou rotaci uzlů v AVL stromu.

\begin{listing}[H]
  \begin{minted}{scala}
  def rotateLeftRight: Node = {
    require(hasBinarySearchTreeStructure)
    require(hasAVLTreeStructure)
    require(left.isAVLTree && right.isAVLTree)
    require(balanceFactor == -2 && left.balanceFactor == 1)

    left match {
      case Node(v, ll, lr, _) =>
        lr match {
          case Node(vlr, lrl, lrr, _) =>
            Node(
              vlr,
              Node(
                v, ll, lrl,
                1 + int_max(ll.height, lrl.height)
              ),
              Node(
                value, lrr, right,
                1 + int_max(lrr.height, right.height)
              ),
              1 + int_max(
                1 + int_max(ll.height, lrl.height),
                1 + int_max(lrr.height, right.height)
              )
            )
        }
    }
  }.ensuring(res => res.content == content && res.isAVLTree)
  \end{minted}
  \caption{Funkce pro levou-pravou rotaci uzlů v AVL stromu}
  \label{lst:stainless-avl-rotateLeftRight}
\end{listing}

\begin{listing}[H]
  \begin{minted}{scala}
  def rotateRightLeft: Node = {
    require(hasBinarySearchTreeStructure)
    require(hasAVLTreeStructure)
    require(left.isAVLTree && right.isAVLTree)
    require(balanceFactor == 2 && right.balanceFactor == -1)

    right match {
      case Node(v, rl, rr, _) =>
        rl match {
          case Node(vrl, rll, rlr, _) =>
            Node(
              vrl,
              Node(
                value, left, rll,
                1 + int_max(left.height, rll.height)
              ),
              Node(
                v, rlr, rr,
                1 + int_max(rlr.height, rr.height)
              ),
              1 + int_max(
                1 + int_max(left.height, rll.height),
                1 + int_max(rlr.height, rr.height)
              )
            )
        }
    }
  }.ensuring(res => res.content == content && res.isAVLTree)
  \end{minted}
  \caption{Funkce pro pravo-levou rotaci uzlů v AVL stromu}
  \label{lst:stainless-avl-rotateRightLeft}
\end{listing}

Poslední funkce \texttt{balance} z výpisu~\ref{lst:stainless-avl-balance}
provádí obecně opravu vyváženosti uzlu.
Rotace uzlů představené v předchozích výpisech
jsou striktně povolené pouze ve specifických případech rozdělených podle
balančního faktoru uzlu.
Obecná funkce \texttt{balance} přijímá k opravě jakýkoli uzel s balančním faktorem
v intervalu [-2, 2], kde pravý i levý strom jsou ale správně vyvážené AVL stromy.

Tento postup byl zvolen z důvodu usnadnění implementace předchozích funkcí \texttt{insert} a \texttt{delete}.
Volání jedné funkce ve včech scénářích, kde je potřeba vyvážit uzel stromu,
je implementačně i důkazově jednodušší než tuto logiku přenášet do zmíněných funkcí,
ve kterých by bylo nutné provést tuto logiku duplicitně.
Tento postup také umožňuje rychlejší dokončení důkazu korektnosti celého programu.

Funkce \texttt{balance} je úmysleně napsaná tak, aby opravila
jakýkoli uzel s nevývážeností v intervalu $[-2, 2]$ nehledě na to,
jak jsou vyvážené jeho podstromy.
Například operace \texttt{insert} nikdy nevygeneruje stav,
ve kterém by rodičovský uzel měl balanční faktor $2$ nebo $-2$
a zároveň potomek, ve kterém proběhlo přidání, tohoto uzlu měl balanční faktor $0$~\cite{Pruvodce22}.
Nicméně pro důkazní popis je tato situace,
kdy něco nemůže nastat, komplikovaně popsatelná.
Proto byla zvolena implementace této funkce tak,
aby se při zavolání \texttt{balance} na uzlu s balančním faktorem $2$ a balančním faktorem pravého syna $0$
zvolila levá rotace.
Výsledek rotace bude totiž AVL strom a jeho obsah zůstane stejný.
Případ balančního faktoru $-2$ a levém synovi s balančním faktorem $0$ je podobný,
pouze s pravou rotací.

\begin{listing}[H]
  \begin{minted}{scala}
  def balance: Node = {
    require(hasBinarySearchTreeStructure)
    require(hasAVLTreeStructure)
    require(left.isAVLTree && right.isAVLTree)
    require(-2 <= balanceFactor && balanceFactor <= 2)

    if (balanceFactor == 2) {
      if (right.balanceFactor == -1) rotateRightLeft
      else rotateLeft
    } else if (balanceFactor == -2) {
      if (left.balanceFactor == 1) rotateLeftRight
      else rotateRight
    } else this
  }.ensuring(res => res.content == content && res.isAVLTree)
  \end{minted}
  \caption{Funkce pro opravu vyváženosti uzlů v AVL stromu}
  \label{lst:stainless-avl-balance}
\end{listing}
